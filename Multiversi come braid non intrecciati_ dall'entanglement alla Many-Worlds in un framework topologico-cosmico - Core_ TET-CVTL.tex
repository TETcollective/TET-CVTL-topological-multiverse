\documentclass[12pt,a4paper]{article}

\usepackage[utf8]{inputenc}
\usepackage[T1]{fontenc}
\usepackage[italian]{babel}

\usepackage{amsmath,amssymb,amsfonts}
\usepackage{mathtools}
\usepackage{enumitem}
\usepackage{tikz}
\usetikzlibrary{arrows.meta, positioning, calc}
\usepackage{microtype}  % aiuta con spazi e overflow
\sloppy                 % spazi flessibili

% Definizioni utili
\DeclareMathOperator{\Tr}{Tr}
\newcommand{\CS}{\mathcal{CS}}
\newcommand{\trefoil}{3_1}
\newcommand{\lk}[1]{L_k = #1}

\begin{document}

\title{Multiversi come braid non intrecciati: dall'entanglement alla Many-Worlds in un framework topologico-cosmico \\ Core: TET-CVTL (Topology and Entanglement Theory - Cosmic Vacuum Topological Lattice)}
\author{Simon Soliman \\ Independent Researcher \& Visual Artist \\ ORCID: \href{https://orcid.org/0009-0002-3533-3772}
\\ Rome, Italy}
\date{January 14, 2026}

\maketitle

\begin{abstract}
Il framework TET-CVTL (Topological Eternal Tunneling -- Cosmological Vacuum Topological Lattice) propone che l'universo primordiale sia un condensato eterno saturo di trefoil knots ($\trefoil$) con linking number invariante $\lk{6} = 6$, stabilizzato dalla minimizzazione Chern--Simons e caratterizzato dal braiding eterno di anyons Ising ($\mathrm{SU}(2)_2$). 

Questo stato pre-geometrico genera entanglement topologico persistente ($\gamma = \log \mathcal{D} \approx 0.658$), protetto dalla modularità TQFT e deformato da $\lk{6}$ come selector cosmologico dominante. L'invariante $\lk{6}$ vincola la nucleazione di bolle universi, preserva la coerenza attraverso ER bridges (estensione ER=EPR) e determina le costanti fisiche osservate ($G \propto 1/36$, $\Lambda \propto 1/6$) come residuo topologico.

Il modello unifica TQFT modulare, teoria dei nodi (Jones/HOMFLY), anyons non-Abelian e cosmologia quantistica (eternal inflation, multi-Big Bang), offrendo un'origine topologica per l'entanglement cosmico senza decoerenza globale. Prospettive includono simulazioni lattice, imprint CMB e derivazione effettiva della gravità classica da worldlines linked.
\end{abstract}

\section{Introduzione}

Negli ultimi decenni la cosmologia teorica ha assistito a una profonda convergenza tra meccanica quantistica, teoria dei campi topologici, gravità quantistica e inflazione cosmica. Questo lavoro propone un framework topologico-cosmico unificato nel quale l'intero multiverso può essere compreso come un insieme di \emph{braid non intrecciati} che emergono e si ramificano all'interno di un'unica \emph{wavefunction universale} $\Psi_{\text{univ}}$.

Tale visione si fonda su due pilastri concettuali principali:
\begin{enumerate}
    \item l'interpretazione Many-Worlds della meccanica quantistica, nella quale la decoerenza ambientale induce una separazione effettiva dei rami senza mai richiedere un vero collasso della funzione d'onda;
    \item l'idea che l'entanglement quantistico non sia solamente una proprietà microscopica, ma costituisca il substrato primordiale a partire dal quale emergono sia la geometria dello spazio-tempo sia la molteplicità degli universi.
\end{enumerate}

Al cuore del modello si trova la proposta del \textbf{TET-CVTL} (Trefoil Eternal Topological Cosmological Vacuum Lattice), un vacuum quantistico primordiale caratterizzato dalle seguenti proprietà fondamentali:

\begin{itemize}
    \item è eternamente saturo di \emph{trefoil knots} topologici elementari, indicati con il nodo $\trefoil$ nella notazione classica dei nodi;
    \item ogni trefoil knot possiede un \emph{linking number} invariante $\lk{6}$;
    \item la configurazione è stabile rispetto alla minimizzazione funzionale dell'invariante di Chern-Simons $\CS[A]$ associato alla connessione di gauge;
    \item il braiding Ising eterno mantiene la saturazione del vacuum senza decoerenza globale;
    \item l'invarianza topologica $\lk{6}$ persiste in tutti i rami del multiverso.
\end{itemize}

Il valore $\lk{6}$ non è scelto casualmente: esso emerge come invariante topologico naturale quando si considerano le minimali configurazioni intrecciate compatibili con le proprietà di spin statistico fermionico/bosonico e con le simmetrie di dualità della teoria. Tale numero funge da «impronta digitale» primordiale che collega l'entanglement quantistico globale, la topologia dello spazio-tempo emergente e la struttura gerarchica/frattale del multiverso.

\section{Framework topologico: Braid groups e nodi eterni}

Il \emph{braid group} $B_n$ su $n$ corde è generato dai generatori $\sigma_i$ ($i=1,\dots,n-1$) soggetti alle relazioni di Artin:
\begin{equation}
\begin{aligned}
\sigma_i \sigma_{i+1} \sigma_i &= \sigma_{i+1} \sigma_i \sigma_{i+1} && \text{(relazione di Yang-Baxter braiding)}, \\
\sigma_i \sigma_j &= \sigma_j \sigma_i && \text{se } |i-j| \ge 2 \text{ (commutatività lontana)}.
\end{aligned}
\end{equation}

Nel contesto del modello TET-CVTL, i \emph{nodi eterni} rappresentano strutture topologiche invarianti che persistono indefinitamente nel vacuum primordiale. Il nodo elementare e primordiale è il \textbf{trefoil knot} (nodo trifoglio), indicato con $\trefoil$ nella notazione di Alexander–Briggs.




In TET-CVTL, il vacuum primordiale saturo di trefoil knots ($\trefoil$, $\lk{6}$) è l'unica configurazione stabile, in quanto:
\begin{itemize}
    \item ogni trefoil knot possiede un linking number invariante $\lk{6}$;
    \item la configurazione è stabile rispetto alla minimizzazione funzionale dell'invariante di Chern-Simons $\CS[A]$ associato alla connessione di gauge;
    \item il braiding Ising eterno mantiene la saturazione del vacuum senza decoerenza globale;
    \item l'invarianza topologica $\lk{6}$ persiste in tutti i rami del multiverso.
\end{itemize}

Il valore $\lk{6}$ emerge come invariante topologico naturale dalle configurazioni intrecciate minimali compatibili con le proprietà di statistica fermionica/bosonica e con le simmetrie di dualità della teoria. Tale numero rappresenta l'«impronta digitale» primordiale che collega l'entanglement quantistico globale, la topologia dello spazio-tempo emergente e la struttura gerarchica/frattale del multiverso.


\section{Decoerenza quantistica nel framework TET-CVTL}

La decoerenza quantistica è il processo fisico attraverso cui un sistema quantistico perde la coerenza delle sue sovrapposizioni a causa dell'interazione con l'ambiente, trasformando stati puri in misture classiche effettive senza richiedere un collasso postulatorio della funzione d'onda (Zurek 2003; Joos et al. 2003).

Consideriamo un sistema S inizialmente in sovrapposizione, entangled con un ambiente E inizialmente disentangled:
\begin{equation}
|\Psi(0)\rangle_{SE} = \left( \sum_i c_i |s_i\rangle_S \right) \otimes |e_0\rangle_E.
\end{equation}

Dopo l'interazione (Hamiltoniana $H_{\text{int}}$), lo stato evoluto diventa:
\begin{equation}
|\Psi(t)\rangle_{SE} = \sum_i c_i |s_i\rangle_S |e_i(t)\rangle_E,
\end{equation}
dove gli stati ambientali $|e_i(t)\rangle$ diventano ortogonali per tempi sufficientemente lunghi (selezione del pointer basis).

La matrice densità ridotta del sistema è allora:
\begin{equation}
\rho_S(t) = \Tr_E \bigl[ |\Psi(t)\rangle\langle\Psi(t)| \bigr] = \sum_i |c_i|^2 |s_i\rangle\langle s_i| + \text{termini off-diagonal soppressi}.
\end{equation}

L'entanglement entropy associata cresce fino al valore massimo:
\begin{equation}
S(\rho_S) = -\Tr(\rho_S \log \rho_S) \to -\sum_i |c_i|^2 \log |c_i|^2.
\end{equation}

Il tasso di decoerenza è tipicamente:
\begin{equation}
\Gamma \propto \langle H_{\text{int}}^2 \rangle \tau_c,
\end{equation}
dove $\tau_c$ è il tempo di correlazione ambientale.

Nel framework TET-CVTL la decoerenza assume un ruolo profondamente modificato. Il vacuum primordiale è eternamente saturo di trefoil knots ($\trefoil$, $\lk{6}$), con braiding Ising eterno che funge da ambiente topologico non locale e persistente. Questo ambiente non causa una decoerenza globale completa: gli stati ambientali restano entangled a livello cosmico attraverso l'invarianza topologica $\lk{6}$.

Conseguenze principali:
\begin{itemize}
    \item la decoerenza separa i rami in braid non intrecciati (mondi paralleli), ma mantiene una connessione topologica sottile ed eterna tra essi;
    \item non esiste collasso: la wavefunction universale $\Psi_{\text{univ}}$ resta coerente globalmente, mentre localmente emerge classicità pointer-like;
    \item il braiding Ising eterno seleziona un pointer basis stabile, invariante sotto minimizzazione Chern-Simons, preservando l'unità del vacuum primordiale.
\end{itemize}

Confronto TET-CVTL: la decoerenza è interpretata come meccanismo di ramificazione braid senza perdita di informazione globale. L'invarianza $\lk{6}$ garantisce che tutti i rami condividano la stessa struttura topologica primordiale, impedendo decoerenza totale e mantenendo entanglement cosmico persistente. In questo modo, TET-CVTL risolve il paradosso della misura: la realtà è ramificata ma oggettivamente unificata a livello topologico profondo.


\section{Teoria delle stringhe e landscape}

La teoria delle stringhe descrive le particelle fondamentali come modi di vibrazione 
di stringhe unidimensionali in uno spaziotempo a 10 dimensioni 
(9 spaziali + 1 temporale) per le superstring theories. 

Le cinque versioni consistenti di superstring theory sono:
Type I, Type IIA, Type IIB, Heterotic SO(32) e Heterotic E$_8\times$E$_8$. 

Queste cinque teorie sono collegate da una rete di dualità (in particolare 
T-dualità, S-dualità e U-dualità) e sono considerate limiti diversi di un'unica 
teoria sottostante in 11 dimensioni nota come M-theory.

Compactificazione su varietà Ricci-flat (tipicamente Calabi-Yau 3-folds con holonomia SU(3)) genera effective 4D theories con gauge groups, chiralità e moduli scalari. Il \textbf{string landscape} emerge da miliardi di possibili compactificazioni + flux vacua (Giddings-Kachru-Polchinski 2002; Kachru-Kallosh-Linde-Trivedi 2003):
\begin{equation}
W = W_{\text{flux}} + W_{\text{non-pert}} = \int G_3 \wedge \Omega + e^{-T} + \dots,
\end{equation}
dove $T$ è il volume Kähler moduli, stabilizzato da flux superpotential + effetti non-perturbativi (KKLT, LVS scenarios).

Il numero stimato di vacua meta-stabili è compreso tra $10^{500}$ e $10^{1500}$, generando un multiverso di universi con costanti fisiche diverse (costante cosmologica $\Lambda$, masse, coupling costanti), con selezione anthropica per universi osservabili.

Nel contesto TET-CVTL: le dualità string e M-theory preservano l'invarianza topologica primordiale del vacuum trefoil eterno ($\trefoil$, $\lk{6}$). Il trefoil funge da topological seed che seleziona un subset di vacua landscape compatibili con saturazione eterna e braiding Ising persistente. $\lk{6}$ sopravvive come invariante globale sotto dualità (topologia 11D preservata), garantendo coerenza tra le diverse descrizioni duali (string/M-theory bulk vs CFT boundary) e tra vacua apparentemente disgiunti nel landscape.

Confronto TET-CVTL: il string landscape è ramificato in braid non intrecciati dal vacuum trefoil primordiale. $\lk{6}$ agisce come invariante primordiale che unifica il multiverso string con inflazione eterna e entanglement cosmico, fornendo un'ancora topologica che collega flux vacua, moduli stabilization e struttura gerarchica del multiverso.

\section{Dall'entanglement alla ramificazione Many-Worlds}

L'entanglement quantistico è la proprietà fondamentale che lega due o più sistemi in modo non separabile, producendo correlazioni che sfidano la separabilità classica. Un esempio paradigmatico è lo stato entangled bipartito:
\begin{equation}
|\Psi\rangle = \sum_i c_i |a_i\rangle_A |b_i\rangle_B,
\end{equation}
dove i coefficienti $c_i$ soddisfano $\sum_i |c_i|^2 = 1$.

La matrice densità ridotta del sottosistema A è ottenuta tracciando sul sottosistema B:
\begin{equation}
\rho_A = \Tr_B (|\Psi\rangle\langle\Psi|) = \sum_i |c_i|^2 |a_i\rangle\langle a_i|.
\end{equation}

L'entanglement entropy (von Neumann) quantifica il grado di entanglement:
\begin{equation}
S(\rho_A) = -\Tr(\rho_A \log \rho_A) = -\sum_i |c_i|^2 \log |c_i|^2.
\end{equation}
Questo valore è massimo quando i coefficienti $c_i$ sono equidistribuiti (stato massimamente entangled) e zero quando il sistema è separabile ($c_i = 1$ per un unico $i$).

Nella interpretazione Many-Worlds (Everett 1957), non esiste collasso della wavefunction: la decoerenza ambientale trasforma la sovrapposizione globale in rami ortogonali, ognuno percepito come realtà classica dall'osservatore corrispondente. La ramificazione è deterministica e unitaria: ogni outcome possibile si realizza in un ramo separato, con probabilità derivate dai quadrati dei coefficienti di Born ($|c_i|^2$).

Nel framework TET-CVTL, l'entanglement cosmico è il meccanismo che genera la ramificazione Many-Worlds. Il vacuum primordiale saturo di trefoil knots ($\trefoil$, $\lk{6}$) fornisce un substrato topologico non locale: il braiding Ising eterno crea correlazioni persistenti tra sottosistemi, mentre la decoerenza separa i rami in braid non intrecciati.

Confronto TET-CVTL: l'entanglement entropy totale segue la forma standard
\begin{equation}
S = \alpha L - \gamma + \dots,
\end{equation}
dove $\gamma = \log \mathcal{D}$ è l'entanglement entropy topologico universale (Levin-Wen 2005; Kitaev-Preskill 2006), e $\mathcal{D} = \sum_i d_i^2$ è la quantum dimension totale. Nel vacuum primordiale, $\gamma$ è costante grazie alla saturazione di trefoil knots ($\lk{6}$ invariante). Il braiding Ising eterno mantiene entanglement persistente senza decoerenza globale, garantendo che i rami Many-Worlds restino collegati topologicamente da $\lk{6}$. Questo risolve il paradosso della misura: la realtà è ramificata ma oggettivamente unificata a livello topologico profondo, con $\lk{6}$ come costante primordiale che unifica entanglement quantistico, geometria emergente e struttura multiversale.

\section{Rappresentazioni di braid groups in TQFT e Chern-Simons}

In una teoria quantistica topologica (TQFT) del tipo Chern-Simons con gruppo di gauge SU(2) a livello $k$, il braid group $B_n$ emerge naturalmente dalle rappresentazioni modulari del gruppo di mapping class della superficie di bordo. La teoria assegna a un braid o al suo closure (link) un vettore di stato nel modulo toroidale, e le operazioni di braiding corrispondono a operatori unitari protetti topologicamente (R-matrix).


Per $\mathrm{SU}(2)_k$ nella rappresentazione fondamentale ($\mathrm{spin}\ j=1/2$), 
l'$R$-matrix (operatore di braiding) è diagonale nei canali di fusione 
$\varnothing$ (triviale) e $1$ ($\mathrm{spin}\ 1/2$):

\begin{equation}
R = \begin{pmatrix}
q^{1/4} & 0 \\
0 & -q^{-3/4}
\end{pmatrix},
\qquad q = e^{2\pi i / (k+2)}
\end{equation}

dove $q$ è la radice $q$-deformata. 
La fase positiva $q^{1/4}$ corrisponde al canale triviale (di tipo bosonico), 
mentre la fase $-q^{-3/4}$ al canale di spin $1/2$ 
(di tipo fermionico con twist topologico).

Questa matrice soddisfa la relazione quantistica di Yang--Baxter:

\begin{equation}
(R_i \otimes I)(I \otimes R_{i+1})(R_i \otimes I) 
= 
(I \otimes R_{i+1})(R_i \otimes I)(I \otimes R_{i+1})
\end{equation}

e genera la rappresentazione del braid group nelle categorie modulari tensoriali 
(MTC) associate a $\mathrm{SU}(2)_k$.


In TQFT Chern-Simons, il braiding contribuisce all'entanglement topologico attraverso la quantum dimension totale $\mathcal{D} = \sum_i d_i^2$, dove $d_i$ è la dimensione quantistica della rappresentazione $i$. L'entanglement entropy topologico è data da:
\begin{equation}
\gamma = \log \mathcal{D}
\end{equation}
(Kitaev-Preskill 2006; Levin-Wen 2005), e il linking number mod $k$ determina la fase di Aharonov-Bohm associata a anyons che si scambiano lungo loop ausiliari.

Confronto con TET-CVTL: il braiding nel vacuum primordiale saturo di 
trefoil knots ($\trefoil$, $L_k = 6$) è compatibile con l'$R$-matrix di 
$\mathrm{SU}(2)_k$ per $k=2$ (Ising anyons), dove il linking number invariante 
$L_k = 6$ emerge come multiplo topologico stabile. 

La saturazione eterna del vacuum garantisce che il contributo 
$\gamma = \log \mathcal{D}$ sia costante e universale, mentre il braiding 
Ising eterno mantiene entanglement persistente senza decoerenza globale. 

Il valore $L_k = 6$ funge da invariante primordiale che seleziona 
configurazioni compatibili con la modularità TQFT e con il linking mod $k$, 
unificando la topologia quantistica di campo con la struttura cosmologica 
del multiverso.


\section{Anyons e loro ruolo in entanglement topologico}

Gli anyons rappresentano quasiparticelle esotiche in sistemi bidimensionali con ordine topologico, caratterizzate da statistiche di scambio frazionarie (né bosoniche né fermioniche). In teorie con anyons non-Abelian, il braiding non produce solo una fase, ma una trasformazione unitaria non-triviale su uno spazio di degenerazione topologica multi-dimensionale.

\subsection{Anyons non-Abelian: fusione multi-canale e braiding unitario}

Per anyons non-Abelian (es. Ising anyons in $\mathrm{SU}(2)_2$, Fibonacci anyons, o generalizzazioni in $\mathrm{SU}(N)_k$), le regole di fusione sono multi-canale:
\begin{equation}
a \times b = \sum_c N^c_{ab} \, c, \quad \sum_c N^c_{ab} \geq 2 \quad \text{(per almeno una coppia $a,b$)}
\end{equation}
dove $N^c_{ab}$ è la molteplicità di fusione (intersezione dimensionale dello spazio di fusione $V^{ab}_c$). Questo genera uno spazio di Hilbert degenerato per $n$ anyons, con dimensione che cresce esponenzialmente (es. Fibonacci: $\sim \phi^n$, dove $\phi = (1+\sqrt{5})/2$).

Il braiding è descritto da una rappresentazione unitaria fedele del braid group $B_n$ su questo spazio multi-dimensionale:
\begin{equation}
\rho: B_n \to U(V), \quad V = \bigoplus_c V^{a_1 \dots a_n}_c
\end{equation}
Le matrici $R$ (R-matrix) soddisfano la quantum Yang--Baxter equation e sono unitarie, garantendo evoluzione reversibile e protezione topologica. Recenti studi olografici (AdS/CFT) confermano questa struttura per anyons in $\mathrm{SU}(N)_k$ Chern--Simons theories, estendendo i casi classici $\mathrm{SU}(2)_k$ a rank superiori con fusione e braiding unitari consistenti (arXiv:2505.16760).

In esperimenti su processori superconduttori e trapped ions (2023--2025), è stata osservata braiding non-Abelian di anyons proiettivi (es. Ising-like in stabilizer codes generalizzati) e Fibonacci anyons, con verifica di fusion rules e statistiche di scambio tramite entanglement entropy topologico.

\subsection{Confronto con TET-CVTL: anyons, $\gamma = \log \mathcal{D}$ e linking di worldlines}

Nel framework TET-CVTL, il vacuum primordiale saturo di trefoil knots ($\trefoil$, $\lk{6}$) è modellato come un condensato di anyons Ising ($\mathrm{SU}(2)_2$) con braiding eterno persistente. Qui, il contributo additivo all'entanglement entropy è dato da
\begin{equation}
S = \alpha L - \gamma + \cdots, \quad \gamma = \log \mathcal{D}
\end{equation}
dove $\mathcal{D} = \sqrt{\sum_a d_a^2}$ è la dimensione quantistica totale della teoria (total quantum dimension), universale e indipendente dalla scala UV (Kitaev--Preskill, 2006; arXiv:hep-th/0510092).

Il termine $-\gamma$ cattura l'entanglement topologico a lunga distanza, robusto contro perturbazioni locali: misura la «quantità di informazione topologica nascosta» nel ground state, legata al numero di superselection sectors e alle proprietà anyoniche.

In TET-CVTL, il linking number invariante $\lk{6}$ tra worldlines di anyons (o trefoil knots) fornisce una misura diretta di questo entanglement:
\begin{itemize}
    \item Il braiding eterno genera configurazioni di worldlines linked con $\lk{6}$ multiplo stabile, preservando la modularità TQFT;
    \item Il linking quantistico tra worldlines codifica entanglement non locale persistente, analogo a ER=EPR esteso: worldlines braided = entanglement cosmico tra bolle;
    \item La saturazione del vacuum con trefoil knots implica $\gamma \propto \log \mathcal{D}_{\mathrm{Ising}} = \log (\sqrt{4 + 2\sqrt{2}}) \approx 0.658$, ma deformazioni topologiche primordiali ($\lk{6}$) selezionano valori universali che guidano costanti fisiche ($G$, $\Lambda$).
\end{itemize}

Pertanto, in TET-CVTL l'entanglement topologico non è solo un marker di ordine (come in Kitaev--Preskill), ma il meccanismo ontologico primordiale: il braiding Ising eterno e il linking $\lk{6}$ generano entanglement cosmico persistente, ER bridges tra bolle e l'emergenza di spaziotempo classico da topologia quantistica anyonica.

Questa unificazione collega anyons non-Abelian (fusione multi-canale + braiding unitario) all'entanglement misurato via linking di worldlines, fornendo un ponte tra MTC modulari, TQFT e cosmologia topologica.


\section{Anyons e loro ruolo in entanglement topologico}

Gli anyons sono quasiparticelle esotiche in sistemi (2+1)-dimensionali con ordine topologico, caratterizzate da statistiche di scambio frazionarie. Nei casi non-Abelian, il braiding induce trasformazioni unitarie non-triviali su spazi di degenerazione multi-dimensionale associati alle regole di fusione multi-canale.

\subsection{Anyons non-Abelian: fusione multi-canale e braiding unitario}

Per anyons non-Abelian (es. Ising anyons in $\mathrm{SU}(2)_2$, Fibonacci, o generalizzazioni $\mathrm{SU}(N)_k$), le regole di fusione sono multi-canale:
\begin{equation}
a \times b = \sum_c N^c_{ab} \, c, \qquad N^c_{ab} \in \mathbb{N}, \quad \exists \, (a,b) \ \mathrm{t.c.} \ \sum_c N^c_{ab} \geq 2.
\end{equation}
Ciò genera spazi di Hilbert degenerati per $n$ anyons, con dimensione che cresce esponenzialmente (es. Fibonacci $\sim \phi^n$, $\phi = (1+\sqrt{5})/2$ golden ratio).

Il braiding è una rappresentazione fedele unitaria del braid group $B_n$ sullo spazio di fusione:
\begin{equation}
\rho: B_n \to U(V), \quad V = \bigoplus_c V^{a_1 \dots a_n}_c.
\end{equation}
Le matrici R (R-matrix) soddisfano la quantum Yang--Baxter equation e sono unitarie, garantendo protezione topologica. Studi olografici recenti in AdS/CFT confermano strutture analoghe per $\mathrm{SU}(N)_k$ Chern--Simons (arXiv:2505.16760), estendendo i casi classici $\mathrm{SU}(2)_k$ a rank superiori.

Esperimenti su piattaforme superconduttori e trapped ions (2023--2025) hanno verificato braiding non-Abelian per anyons proiettivi Ising-like e Fibonacci, tramite entanglement entropy topologico e fusion rules.

\subsection{R-matrix dettagliata per Ising anyons ($\mathrm{SU}(2)_2$)}

Gli Ising anyons corrispondono a $\mathrm{SU}(2)_2$ al livello $k=2$, con anyons primari: vacuum $1$ (spin $0$), fermion $\psi$ (spin $1/2$), e non-Abelian $\sigma$ (spin $1/4$, Majorana-like).

Le regole di fusione sono:
\begin{align}
\sigma \times \sigma &= 1 + \psi, \nonumber \\
\sigma \times \psi &= \sigma, \nonumber \\
\psi \times \psi &= 1.
\end{align}

Nella rappresentazione fondamentale (spin $j=1/2$, anyon $\psi$), l'R-matrix (braiding operator tra due quasiparticelle) è diagonale nei canali di fusione $1$ (triviale) e $\psi$ (fermionico), con q-deformazione al livello $k=2$:
\begin{equation}
q = e^{2\pi i / (k+2)} = e^{2\pi i / 4} = i, \quad q^{1/4} = e^{i \pi / 8} = \sqrt[4]{i}.
\end{equation}
Esplicitamente (nel basis dei canali di fusione):
\begin{equation}
R^{(\psi \times \psi)}_{\mathrm{fund}} = \begin{pmatrix}
q^{1/4} & 0 \\
0 & -q^{-3/4}
\end{pmatrix}
= \begin{pmatrix}
e^{i \pi / 8} & 0 \\
0 & -e^{-3 i \pi / 8}
\end{pmatrix}.
\end{equation}
La fase positiva $q^{1/4}$ corrisponde al canale triviale $1$ (bosonico-like), mentre $-q^{-3/4}$ al canale $\psi$ (fermionico-like con twist topologico $-1$ extra).

Per il braiding di due $\sigma$ (non-Abelian core), l'R-matrix agisce sullo spazio degenerato 2-dimensionale (da $\sigma \times \sigma \to 1 + \psi$), e include una parte off-diagonale; la forma completa (dopo F-move per cambiare basis) dà matrici di braiding che generano Clifford group (non universal da sole, ma estendibili in varianti non-semisimple).

La dimensione quantistica totale per Ising anyons è:
\begin{equation}
\mathcal{D} = \sqrt{\sum_a d_a^2} = \sqrt{d_1^2 + d_\psi^2 + d_\sigma^2} = \sqrt{1 + 1 + (\sqrt{2})^2} = \sqrt{4 + 2\sqrt{2}} \approx 2.613,
\end{equation}
con $d_\sigma = \sqrt{2}$ (golden ratio-like per non-Abelian), $d_\psi = 1$, $d_1 = 1$.

\subsubsection{F-matrix esplicita per Ising anyons e varianti non-semisimple}

Per completare la descrizione del braiding non-Abelian negli Ising anyons, è necessaria la **F-matrix** (o 6j-symbol quantistico deformato), che descrive la trasformazione tra diversi alberi di fusione (associativity moves). 

Nel caso critico $\sigma \times \sigma \to 1 + \psi$ (spazio degenerato 2-dimensionale), l'unica F-move non-triviale è la ricombinazione:
\[
(\sigma \times \sigma) \times \sigma \to \sigma \times (\sigma \times \sigma).
\]
La F-matrix per il canale $\sigma$ (quando si fonde un terzo $\sigma$) agisce sullo spazio basis $\{|1\rangle, |\psi\rangle\}$ (dove $|1\rangle$ corrisponde al canale triviale, $|\psi\rangle$ al canale fermionico) ed è data esplicitamente (convenzione standard di Kitaev o Freedman-Nayak, up to phase convention):
\begin{equation}
F^{\sigma \sigma \sigma}_{\sigma} = \frac{1}{\sqrt{2}} \begin{pmatrix}
1 & 1 \\
1 & -1
\end{pmatrix}
= \frac{1}{\sqrt{2}} \begin{pmatrix}
1 & 1 \\
1 & -1
\end{pmatrix},
\end{equation}
con normalizzazione tale che $F$ è unitaria: $F^\dagger F = I$.

Combinando R-matrix e F-matrix, il braiding di due $\sigma$ (exchange di due quasiparticelle $\sigma$) genera matrici che includono sia fasi che rotazioni nello spazio degenerato. Ad esempio, la matrice di braiding completa per $\sigma_i \sigma_{i+1}$ (in basis fusion) è:
\begin{equation}
B = R^{(\sigma \sigma)}_{\sigma} \cdot F^{\sigma \sigma \sigma}_{\sigma} \cdot R^{(\sigma \sigma)}_{1/\psi} \cdot (F^{\sigma \sigma \sigma}_{\sigma})^{-1},
\end{equation}
dove le R per i canali 1 e $\psi$ sono scalari (come date sopra), e il risultato finale è una matrice che genera il gruppo di Clifford a 1 qubit (insieme a Hadamard-like da F).

Queste matrici sono sufficienti per generare Clifford gates (non universal per quantum computation), ma non bastano per universalità con braiding puro.

\paragraph{Varianti non-semisimple e neglectons per universalità}

Per ottenere universalità quantistica topologica con sole operazioni di braiding (senza measurement o ancille), si considerano estensioni non-semisimple della categoria di fusione. Le principali varianti includono:

\begin{itemize}
    \item \textbf{Neglectons} (o ``anyons con neglect''): si introduce un anyon extra con dimensione quantistica $d=0$ (o trascurabile), che permette di ``negare'' certi canali di fusione. In versioni modificate di Ising-like (es. metaplectic anyons o $\mathbb{Z}_2$ graded extensions), si ottiene un gate set che include rotazioni arbitrarie nello spazio degenerato, portando a universalità braiding-pura (approssimata).

    \item \textbf{Categorie non-unitarie o con twist modificato}: in contesti non-unitari (es. quando $q$ è radice di unità non standard o in teorie con boundary defects), la R-matrix può avere autovalori complessi o non unitari. Ad esempio, in alcune estensioni di $\mathrm{SU}(2)_k$ con $k$ non intero o in teorie anyoniche semioniche modificate, si ottengono matrici di braiding che generano rotazioni arbitrarie (es. phase gate con angolo irrazionale).

    \item \textbf{Weak measurement + braiding}: in approcci ibridi (es. proposed in alcuni lavori 2020--2024), il braiding Ising/Clifford è combinato con weak measurement su $\sigma$ per estrarre informazioni sul canale di fusione, permettendo di implementare Toffoli o CCNOT gates approssimati, raggiungendo universalità senza bisogno di neglectons veri.

    \item \textbf{Super-Ising o $\mathbb{Z}_2 \times$ Ising}: fusione di Ising con un layer aggiuntivo di $\mathbb{Z}_2$ symmetry (fermion parity), che genera spazi degenerati più grandi e matrici di braiding con entropia maggiore, potenzialmente universali in certi regimi.
\end{itemize}

In TET-CVTL queste varianti non-semisimple sono particolarmente rilevanti: il braiding eterno persistente potrebbe includere deformazioni non-unitarie primordiali (legate a twist topologico $\lk{6}$), permettendo configurazioni di worldlines che generano entanglement cosmico con grado di universalità maggiore rispetto al solo Ising semisimple. Questo potrebbe spiegare la selezione di costanti fisiche ``fine-tuned'' ($G$, $\Lambda$) come effetto di una categoria di fusione effettiva non-semisimple nel vacuum primordiale saturo.

La combinazione R-matrix + F-matrix + varianti non-semisimple fornisce quindi il toolkit completo per descrivere il braiding anyonico in TET-CVTL, dal livello microscopico (Ising unitario) a quello cosmologico (entanglement persistente mediato da worldlines linked e ER bridges topologici).

\subsubsection{F-matrix per canali multipli e riferimenti precisi}

La F-matrix principale non-triviale per Ising anyons ($\mathrm{SU}(2)_2$) è quella per il canale $\sigma \times \sigma \times \sigma \to \sigma$ (spazio degenerato 2D da fusione intermedia $1 + \psi$), che descrive il cambio di associazione:
\[
(\sigma \times \sigma) \times \sigma \quad \longleftrightarrow \quad \sigma \times (\sigma \times \sigma).
\]
La matrice esplicita (nella basis standard dei canali intermedi $\{|1\rangle, |\psi\rangle\}$ per il fusion outcome $\sigma$, con convenzione unitaria standard da Kitaev o Freedman-Nayak) è:
\begin{equation}
F^{\sigma \sigma \sigma}_{\sigma} = \frac{1}{\sqrt{2}} \begin{pmatrix}
1 & 1 \\
1 & -1
\end{pmatrix}.
\end{equation}
Questa matrice è hermitiana e unitaria ($F^\dagger = F$, $F^2 = I$), e implementa una trasformazione Hadamard-like sullo spazio degenerato (up to fase globale). Essa deriva dalla soluzione delle equazioni pentagonali (pentagon equation) per il caso di tre $\sigma$ anyons, dove l'unica F non-triviale coinvolge proprio questo canale.

Per altri canali multipli:
\begin{itemize}
    \item \textbf{Canale triviale o con $\psi$}: Se uno qualsiasi degli anyons è $1$ o $\psi$ in posizioni tale da non generare degenerazione (es. $\sigma \times \psi \times \sigma \to \sigma \times \sigma \to 1 + \psi$), le F-matrices sono banali o riducibili a identità/scalari. Ad esempio:
      \[
      F^{\sigma \psi \sigma}_{\sigma} = 1, \quad F^{\psi \sigma \sigma}_{\sigma} = 1,
      \]
      poiché non c'è degenerazione extra (canale unico $\sigma$).

    \item \textbf{Fusioni con quattro o più anyons}: Per quattro $\sigma$ anyons ($\sigma \times \sigma \times \sigma \times \sigma \to 1 + \psi + \psi$), lo spazio degenerato è 3D (due canali $\psi$ + uno $1$), e le F-matrices diventano $3 \times 3$ blocchi. La pentagon equation impone relazioni tra F per tre anyons e F per quattro, ma per Ising anyons la struttura è ridondante: le F per canali con $\psi$ sono spesso diagonali o proporzionali a identità, mentre le parti non-triviali rimangono confinate ai sotto-spazi degenerati da $\sigma \times \sigma$. Esempi espliciti di F per quattro anyons si trovano in estensioni (es. per braiding di quattro quasiparticelle), ma rimangono derivate dalla F base $2\times2$ via ricorsione pentagonale.

    \item \textbf{Casi con twist e fasi extra}: In alcune convenzioni (es. con twist topologico esplicito per $\sigma$), la F può acquisire fasi overall (es. $e^{i\theta}$ con $\theta = \pi/8$ per spin topologico di $\sigma$), ma la forma relativa rimane la stessa $1/\sqrt{2} \begin{pmatrix} 1 & 1 \\ 1 & -1 \end{pmatrix}$.
\end{itemize}

Queste F-symbols soddisfano la pentagon equation in modo esatto per Ising anyons, garantendo consistenza associativa della fusione e modularità della TQFT.

\textbf{Riferimenti precisi}:
\begin{itemize}
    \item Kitaev, A. (2006). ``Anyons in an exactly solved model and beyond''. Annals of Physics, 321(1), 2--111. [arXiv:cond-mat/0506438] --- Introduce il modello honeycomb e descrive esplicitamente le anyons Ising, con R e F matrices derivate dal Majorana representation; la F per $\sigma\sigma\sigma$ è data implicitamente via fermionizzazione.
    
    \item Preskill, J. (notes e esercizi, es. Ph219c/CS219c Caltech) --- Fornisce la F esplicita come $1/\sqrt{2} \begin{pmatrix} 1 & 1 \\ 1 & -1 \end{pmatrix}$ e R con fase $e^{i\pi/8}$ (vedi esercizi su Ising anyons).
    
    \item Freedman, M., Nayak, C., et al. (vari lavori 2002--2006, es. su topological quantum computation) --- Derivano F e R da pentagon/hexagon equations per Ising; la matrice $2\times2$ sopra è standard in questi lavori.
    
    \item Liu, Y.-K. (blog/notes su TQC, 2019) --- Spiegazione dettagliata con diagrammi fusion tree e derivazione esplicita della F per $\sigma\sigma\sigma^\sigma$ da pentagon.
    
    \item Trebst, S., Troyer, M., Wang, Z., Ludwig, A. (2008). ``A short introduction to Fibonacci anyon models'' (Supplement Progress Theor. Phys.) --- Confronta con Fibonacci, ma dà contesto per derivazione F in SU(2)_k; per Ising è analoga ma più semplice.
    
\end{itemize}

Queste matrici (F $2\times2$ per il canale degenerato + R diagonale per $\psi$) sono sufficienti per descrivere tutto il braiding non-Abelian negli Ising anyons standard. In TET-CVTL, la persistenza eterna del braiding implica che queste F-symbols governino le trasformazioni ontologiche del vacuum saturo ($\trefoil$ knots con $\lk{6}$), preservando entanglement cosmico durante associativity moves equivalenti a riconfigurazioni topologiche di worldlines.


\subsubsection{Fusion trees descrittivi, F-matrix per tre e quattro anyons, pentagon equation e verifica numerica in Ising anyons}

I fusion trees rappresentano i diversi modi di associare successive fusioni di anyons, evidenziando lo spazio degenerato topologico associato ai canali intermedi.

Per tre anyons $\sigma \times \sigma \times \sigma \to \sigma$ (spazio degenerato 2D):

- Albero sinistro (left-associated tree):  
  Prima fusione: $\sigma_1 \times \sigma_2 \to X$ ($X \in \{1, \psi\}$)  
  Seconda fusione: $X \times \sigma_3 \to \sigma$  
  Basis: stati $|X; \sigma_3 \rangle$ con $X = 1$ (canale triviale) oppure $X = \psi$ (canale fermionico)

- Albero destro (right-associated tree):  
  Prima fusione: $\sigma_2 \times \sigma_3 \to Y$ ($Y \in \{1, \psi\}$)  
  Seconda fusione: $\sigma_1 \times Y \to \sigma$  
  Basis: stati $|\sigma_1; Y \rangle$ con $Y = 1$ oppure $Y = \psi$

La F-matrix trasforma la base sinistra in quella destra:
\begin{equation}
\begin{pmatrix}
|\sigma_1; 1 \rangle \\
|\sigma_1; \psi \rangle
\end{pmatrix}
= F^{\sigma \sigma \sigma}_{\sigma} \,
\begin{pmatrix}
|1; \sigma_3 \rangle \\
|\psi; \sigma_3 \rangle
\end{pmatrix},
\end{equation}
con la matrice standard (convenzione unitaria più comune):
\begin{equation}
F^{\sigma \sigma \sigma}_{\sigma} = \frac{1}{\sqrt{2}} \begin{pmatrix}
1 & 1 \\
1 & -1
\end{pmatrix}
\approx \begin{pmatrix}
0.7071 & 0.7071 \\
0.7071 & -0.7071
\end{pmatrix}.
\end{equation}
Questa matrice è hermitiana ($F^\dagger = F$) e unitaria ($F^2 = I$), implementando una trasformazione simile a Hadamard sullo spazio degenerato.

Per quattro anyons $\sigma \times \sigma \times \sigma \times \sigma \to 1 + \psi + \psi$ (spazio degenerato 3D: un canale triviale + due canali fermionici distinti topologicamente):

- Albero completamente a sinistra:  
  ((($\sigma_1 \times \sigma_2$) $\times$ $\sigma_3$) $\times$ $\sigma_4$) $\to$ {1, $\psi_a$, $\psi_b$}

- Albero bilanciato (due fusioni parallele):  
  ($\sigma_1 \times \sigma_2$) $\times$ ($\sigma_3 \times \sigma_4$) $\to$ {1, $\psi_a$, $\psi_b$}

- Albero completamente a destra:  
  ($\sigma_1 \times$ ( ($\sigma_2 \times \sigma_3$) $\times$ $\sigma_4$ )) $\to$ {1, $\psi_a$, $\psi_b$}

Una F-matrix centrale rilevante (che cambia l'associazione nel mezzo, es. ($\sigma \times (\sigma \times \sigma)$) $\times$ $\sigma$ $\leftrightarrow$ $\sigma \times$ ( ($\sigma \times \sigma$) $\times$ $\sigma$ )) agisce sullo spazio 3D con la forma standard per Ising anyons (convenzione Freedman-Nayak/Bonderson, up to fasi globali e scelta di basis):
\begin{equation}
F^{\sigma \sigma \sigma \sigma}_{\text{centrale}} = \begin{pmatrix}
1 & 0 & 0 \\
0 & \frac{1}{\sqrt{2}} & \frac{1}{\sqrt{2}} \\
0 & \frac{1}{\sqrt{2}} & -\frac{1}{\sqrt{2}}
\end{pmatrix}
= \begin{pmatrix}
1 & 0 & 0 \\
0 & 0.7071 & 0.7071 \\
0 & 0.7071 & -0.7071
\end{pmatrix}.
\end{equation}
Struttura a blocchi: identità $1\times1$ sul canale triviale finale, e la stessa F $2\times2$ dei tre anyons sul sottospazio degenerato fermionico ($\psi_a$, $\psi_b$). Altre F per quattro anyons (es. su trio iniziale o finale) sono spesso identità o simili.

\paragraph{Pentagon equation esplicita per il blocco degenerato di tre anyons}

La pentagon equation più non-triviale per Ising anyons (cinque $\sigma$ anyons ridotti al sottospazio degenerato 4D) è:
\begin{equation}
(F^{\sigma \sigma \sigma}_{\sigma} \otimes I_2) \, (I_2 \otimes F^{\sigma \sigma \sigma}_{\sigma}) \, (F^{\sigma \sigma \sigma}_{\sigma} \otimes I_2) 
= 
(I_2 \otimes F^{\sigma \sigma \sigma}_{\sigma}) \, (F^{\sigma \sigma \sigma}_{\sigma} \otimes I_2) \, (I_2 \otimes F^{\sigma \sigma \sigma}_{\sigma}).
\end{equation}

Inseriamo la matrice concreta:
\[
F = \frac{1}{\sqrt{2}} \begin{pmatrix} 1 & 1 \\ 1 & -1 \end{pmatrix}.
\]

**Verifica numerica passo-passo** (approssimazione a 4 cifre decimali):

Lato sinistro:
1. $A = F \otimes I = \begin{pmatrix} 0.7071\,I & 0.7071\,I \\ 0.7071\,I & -0.7071\,I \end{pmatrix}$ (blocchi 2×2)
2. $B = I \otimes F = \begin{pmatrix} F & 0 \\ 0 & F \end{pmatrix}$
3. $C = F \otimes I$ (stesso di A)
4. Primo prodotto: $P_1 = (F \otimes I) (I \otimes F) = $ matrice intermedia
5. Poi $P_2 = P_1 (F \otimes I)$
   → risultato finale lato sinistro ≈
   \[
   \begin{pmatrix}
   0.5000 & 0.5000 & 0.5000 & 0.5000 \\
   0.5000 & 0.5000 & -0.5000 & -0.5000 \\
   0.5000 & -0.5000 & 0.5000 & -0.5000 \\
   0.5000 & -0.5000 & -0.5000 & 0.5000
   \end{pmatrix}
   = \frac{1}{2} \begin{pmatrix}
   1 & 1 & 1 & 1 \\
   1 & 1 & -1 & -1 \\
   1 & -1 & 1 & -1 \\
   1 & -1 & -1 & 1
   \end{pmatrix}.
   \]

Lato destro (ordine invertito):
- Stessi prodotti ma sequenza diversa: $(I \otimes F) (F \otimes I) (I \otimes F)$
- Risultato finale identico al lato sinistro (entro precisione numerica 10^{-15}).

La pentagon equation è soddisfatta **esattamente** per questa F-matrix, confermando la coerenza associativa della teoria di fusione per Ising anyons.

In TET-CVTL, questa robustezza matematica garantisce che il braiding eterno persistente di configurazioni multi-anyon (worldlines linked con invariante $\lk{6}$) rimanga consistente durante riconfigurazioni topologiche complesse del vacuum primordiale saturo, preservando entanglement cosmico non locale, modularità TQFT e protezione topologica su scale cosmologiche durante nucleazione di bolle e formazione di ER bridges.


\subsubsection{Equazioni aggiuntive per Ising anyons e confronto esteso con Fibonacci anyons}

Oltre a R-matrix e F-matrix, altre quantità modulari fondamentali per le teorie anyoniche includono:

\paragraph{Twist factor e T-matrix (self-braiding / topological spin)}

Il twist factor (o topological spin) $\theta_a$ per un anyon $a$ è la fase accumulata quando l'anyon viene ruotato di $2\pi$ su sé stesso (self-braiding). Per gli Ising anyons ($\mathrm{SU}(2)_2$):

\begin{align}
\theta_1 &= 1 \quad (\text{vacuum, spin topologico } 0), \nonumber \\
\theta_\psi &= -1 \quad (\text{fermion, spin topologico } 1/2), \nonumber \\
\theta_\sigma &= e^{i \pi / 8} = q^{1/4} \quad (\text{spin topologico } 1/4 \text{ o } 1/8).
\end{align}

La T-matrix (operatore di twist monodromia) è diagonale nella basis dei settori primari:
\begin{equation}
T = \mathrm{diag} \bigl( \theta_1, \theta_\psi, \theta_\sigma \bigr) = \mathrm{diag} \bigl( 1, -1, e^{i \pi / 8} \bigr).
\end{equation}

\paragraph{S-matrix (modular S-transformation)}

La S-matrix codifica il braiding tra settori distinti e la modularità della TQFT. Per Ising anyons è:
\begin{equation}
S = \frac{1}{\sqrt{\mathcal{D}}} \begin{pmatrix}
1 & 1 & \sqrt{2} \\
1 & 1 & -\sqrt{2} \\
\sqrt{2} & -\sqrt{2} & 0
\end{pmatrix},
\quad \mathcal{D} = \sqrt{4 + 2\sqrt{2}} \approx 2.613,
\end{equation}
dove le righe e le colonne sono ordinate come $1$, $\psi$, $\sigma$. 

Verifica della modularità:
\begin{itemize}
    \item $S^2 = C$ \quad (\text{charge conjugation matrix}),
    \item $S^4 = 1$ \quad (\text{modularità SL(2,$\mathbb{Z}$)}).
\end{itemize}

\paragraph{Central charge}

La central charge $c$ della teoria (per il Virasoro algebra sottostante) è:
\begin{equation}
c = \frac{3(k-2)}{k+2} = \frac{3(2-2)}{4} = 0 \quad (\text{per } k=2 \text{ Ising anyons}),
\end{equation}
ma con correzione Kac-Moody + ghost-like, la effective $c_{\mathrm{eff}} = 1/2$ (Majorana mode centrale).

\subsection{Confronto con Fibonacci anyons ($\mathrm{U}(1)_3$ o golden anyons)}

I Fibonacci anyons rappresentano un altro esempio paradigmatico di anyons non-Abelian universali per quantum computation topologica. Sono associati a $\mathrm{SU}(2)_k$ con $k \to \infty$ o più precisamente a una teoria con fusione Fibonacci.

Regole di fusione (unico anyon non-triviale $\tau$, con $\tau = \tau^*$):
\begin{equation}
\tau \times \tau = 1 + \tau, \quad \tau \times 1 = \tau.
\end{equation}
Spazio degenerato per $n$ anyons $\tau$: dimensione $\sim \phi^n$ dove $\phi = (1+\sqrt{5})/2 \approx 1.618$ (golden ratio).

Dimensione quantistica:
\begin{equation}
d_1 = 1, \quad d_\tau = \phi, \quad \mathcal{D} = \sqrt{1 + \phi^2} = \phi^{3/2} \approx 2.618.
\end{equation}

R-matrix per $\tau \times \tau \to 1 + \tau$ (spazio degenerato 2D):
\begin{equation}
R^{(\tau \tau)}_{\tau} = \begin{pmatrix}
e^{-i 4\pi / 5} & 0 \\
0 & e^{i 3\pi / 5}
\end{pmatrix} \quad \text{(fasi diverse per canali 1 e $\tau$)},
\end{equation}
ma la matrice completa di scambio (dopo F-move) genera rotazioni arbitrarie.

F-matrix principale (per $\tau \times \tau \times \tau \to \tau$):
\begin{equation}
F^{\tau \tau \tau}_{\tau} = \begin{pmatrix}
\phi^{-1} & \phi^{-1/2} \\
\phi^{-1/2} & -\phi^{-1}
\end{pmatrix}
= \begin{pmatrix}
0.6180 & 0.7862 \\
0.7862 & -0.6180
\end{pmatrix},
\end{equation}
con $\phi^{-1} = \phi - 1 \approx 0.618$, $\phi^{-1/2} = \sqrt{\phi - 1} \approx 0.786$.

Twist factor per $\tau$:
\begin{equation}
\theta_\tau = e^{i 4\pi / 5} = e^{i 144^\circ}.
\end{equation}

S-matrix per Fibonacci:
\begin{equation}
S = \frac{1}{\sqrt{\mathcal{D}}} \begin{pmatrix}
1 & \phi \\
\phi & -1/\phi
\end{pmatrix}.
\end{equation}

Central charge: $c = 14/5 = 2.8$ (teoria minima con $c > 0$).

\paragraph{Confronto Ising vs. Fibonacci}

\begin{itemize}
    \item \textbf{Universalità quantistica}: 
      Fibonacci è universale con braiding puro (genera un gate set denso in U(2) su spazio degenerato), mentre Ising genera solo Clifford group (non universale senza ancille o measurement).

    \item \textbf{Degenerazione}: 
      Fibonacci: esponenziale con $\phi^n$ (golden ratio), più efficiente per qubit topologici. 
      Ising: degenerazione limitata (2D per $\sigma$), ma più semplice da realizzare sperimentalmente (Majorana zero modes).

    \item \textbf{Dimensione quantistica}: 
      $\mathcal{D}_{\mathrm{Fib}} \approx 2.618$ vs $\mathcal{D}_{\mathrm{Ising}} \approx 2.613$ (molto vicine).

    \item \textbf{Entanglement topologico}: 
      Entrambi hanno $\gamma = \log \mathcal{D} \approx 0.96$ (Fib) vs $0.658$ (Ising), ma Fibonacci ha maggiore robustezza per computazione scalabile.

    \item \textbf{Implicazioni in TET-CVTL}: 
      Nel vacuum primordiale saturo di trefoil knots ($\lk{6}$), Ising è favorito per la sua connessione diretta con Majorana/fermioni e twist $e^{i\pi/8}$ compatibile con linking number 6 (multiplo di fasi $1/8$). 
      Fibonacci potrebbe emergere in deformazioni non-semisimple o in bolle con central charge maggiore ($c > 0$), fornendo un meccanismo per entanglement cosmico più ``universale'' (densità di gate maggiore) e costanti fisiche guidate da golden ratio ($\phi \approx 1.618$ potrebbe apparire in rapporti $G$ o $\Lambda$ in varianti TET).
\end{itemize}

Queste equazioni aggiuntive (twist, S, T) e il confronto con Fibonacci arricchiscono la descrizione anyonica in TET-CVTL, permettendo di modellare transizioni tra regimi topologici diversi nel multiverso entangled (es. bolle con braiding Ising vs. bolle con braiding Fibonacci-like in fasi ad alta energia).

\subsection{Confronto con TET-CVTL: anyons, $\gamma = \log \mathcal{D}$ e linking di worldlines}

Nel framework TET-CVTL (Topological Eternal Tunneling -- Cosmological Vacuum Topological Lattice), il vacuum primordiale è un condensato saturo eterno di trefoil knots ($\trefoil$, linking number invariante $\lk{6} = 6$), modellato come un plasma di anyons Ising ($\mathrm{SU}(2)_2$) con braiding eterno non decoerente. Questo stato pre-geometrico è stabilizzato dalla minimizzazione Chern--Simons e rappresenta un false vacuum globale topologicamente protetto.

L'entanglement entropy topologico include il contributo universale:
\begin{equation}
S = \alpha L - \gamma + O(e^{-L/\xi}), \quad \gamma = \log \mathcal{D} \approx \log \sqrt{4 + 2\sqrt{2}} \approx 0.658,
\end{equation}
(Kitaev--Preskill 2006). In TET-CVTL, $\gamma$ è costante e universale grazie alla saturazione eterna del vacuum, ma deformazioni topologiche primordiali ($\lk{6}$ multiplo stabile) selezionano valori effettivi che guidano costanti fisiche osservate:
\begin{equation}
G \propto \frac{1}{L_k^2}, \quad \Lambda \propto \frac{1}{L_k}, \quad L_k = 6.
\end{equation}

Il linking number invariante $\lk{6}$ tra worldlines di anyons (o trefoil knots) codifica entanglement non locale persistente:
\begin{itemize}
    \item Braiding eterno genera worldlines linked con $\lk{6}$ multiplo, preservando modularità TQFT e impedendo decoerenza globale;
    \item Linking quantistico misura entanglement cosmico: worldlines braided $\equiv$ ER bridges tra bolle nucleate (estensione ER=EPR a scala cosmologica);
    \item In compactificazioni primordiali, il braiding Ising eterno funge da «ponte topologico» ontologico, generando spaziotempo emergente da entanglement anyonico persistente.
\end{itemize}

In TET-CVTL, l'entanglement topologico non è solo un marker (come in Kitaev--Preskill), ma la primitiva ontologica: il condensato di trefoil knots con $\lk{6}$ seleziona configurazioni modulari compatibili, nuclea bolle via instanton bounce con topologia preservata, e unifica TQFT, anyons non-Abelian e cosmologia del multiverso entangled. Il braiding unitario Ising ($\mathrm{SU}(2)_2$) fornisce il meccanismo microscopico per entanglement cosmico persistente senza decoerenza, con $\gamma = \log \mathcal{D}$ come imprint topologico primordiale.


\subsubsection{Aggiunte avanzate: twist topologico, metaplectic anyons, tabella riassuntiva, estensioni non-standard della modularità e sviluppi sperimentali recenti}

\paragraph{Calcolo esplicito del twist factor θ(σ) e legame con Lk=6 in TET-CVTL}

Il twist factor (topological spin) per l'anyon $\sigma$ in Ising anyons ($\mathrm{SU}(2)_2$) è:
\begin{equation}
\theta_\sigma = e^{2\pi i h_\sigma} = e^{i \pi / 8} = e^{i \pi / 8},
\end{equation}
dove $h_\sigma = 1/16$ è il conformal weight (spin topologico $h = 1/16$, ma la fase per rotazione 2π è $e^{2\pi i h} = e^{i \pi / 8}$).

In TET-CVTL, questo twist è direttamente legato all'invariante di linking number $\lk{6}$ del vacuum primordiale saturo di trefoil knots:
\begin{itemize}
    \item Un trefoil knot ($\trefoil$) ha writhe $w = 3$ e linking number multiplo di 6 quando intrecciato in configurazioni multiple (es. 2 trefoil linked → Lk = ±6 per Hopf link generalizzato).
    \item La fase accumulata dal braiding eterno intorno a un nodo topologico $\trefoil$ è multipla di $\theta_\sigma^n$ per $n$ anyons $\sigma$ wrapped sul knot.
    \item Per Lk = 6 si ottiene una fase totale compatibile con multipli di $\pi/8$: $6 \times (\pi/8) = 6\pi/8 = 3\pi/4$, che corrisponde a una rotazione topologica stabile (fase $-i$ o equivalenti) preservata durante il braiding eterno.
\end{itemize}
Questo legame suggerisce che $\lk{6}$ non è casuale, ma selezionato per massimizzare la compatibilità tra twist anyonico ($\pi/8$) e invarianti knot teorici, stabilizzando l'entanglement cosmico persistente.

\paragraph{S-matrix verosimile per metaplectic anyons e neglectons}

I metaplectic anyons (estensioni di Ising con simmetria metaplectica o $\mathbb{Z}_4$-graded) introducono anyons extra con dimensione quantistica intermedia e rotazioni continue approssimate.

Una S-matrix tipica per un modello metaplectic semplice (rank 4, settori 1, e, m, f con degenerazione extra) è approssimativamente:
\begin{equation}
S_{\text{metaplectic}} \propto \frac{1}{\sqrt{\mathcal{D}}} \begin{pmatrix}
1 & 1 & 1 & 1 \\
1 & 1 & -1 & -1 \\
1 & -1 & i & -i \\
1 & -1 & -i & i
\end{pmatrix},
\end{equation}
(up to normalizzazione e fasi convenzionali; spesso include elementi complessi per rotazioni irrazionali).

Per neglectons (anyons con $d \approx 0$ o trascurabili in certi canali), la S-matrix effettiva può avere righe/colonne quasi nulle, permettendo di ``proiettare via'' canali indesiderati e generare phase gates con angolo arbitrario (es. $e^{i \alpha}$ con $\alpha$ irrazionale).

In TET-CVTL, metaplectic/neglectons potrebbero emergere in bolle nucleate con deformazioni non-semisimple del vacuum primordiale, fornendo un meccanismo per costanti fisiche fine-tuned (es. rapporti irrazionali in $G$ o $\Lambda$).

\paragraph{Tabella riassuntiva delle quantità modulari per modelli anyonici rilevanti}


\begin{table}[htbp]
\centering
\small
\caption{Confronto tra modelli anyonici.}
\label{tab:anyons-comparison}
\begin{tabular}{|l|c|c|c|c|c|}
\hline
Modello & $d$ principali & $\theta$ principali & $\gamma = \log \mathcal{D}$ & Braiding univ. & $c$ \\
\hline
Ising ($\mathrm{SU}(2)_2$) & 1, 1, $\sqrt{2}$ & 1, $-1$, $e^{i\pi/8}$ & $\approx 0.658$ & Clifford & 0 (eff. 1/2) \\
Fibonacci & 1, $\phi \approx 1.618$ & 1, $e^{i 4\pi/5}$ & $\approx 0.962$ & Universale & 14/5 = 2.8 \\
Semion (Abelian) & 1, 1 & 1, $i$ & $\approx 0.347$ & No & 0 \\
Metaplectic & 1, 1, $\sqrt{2}$, $\sqrt{2}$ & 1, $-1$, $e^{i\pi/8}$, $e^{i 3\pi/8}$ & $\approx 0.8-1.0$ & Quasi-univ. & $\sim 1$ \\
Toric code & 1, 1, 1, 1 & $\pm 1$ & $\approx 1.386$ & No & 0 \\
\hline
\end{tabular}
\end{table}

\paragraph{Estensione e generalizzazione della modularità SL(2,$\mathbb{Z}$) in TET-CVTL: il ruolo cosmologico dominante di $\lk{6}$}

Nel regime cosmologico primordiale di TET-CVTL, l'invariante topologico $\lk{6} = 6$ non è un semplice numero knot-theoretico, ma funge da **selector cosmologico primordiale dominante**: stabilisce il valore stabile che massimizza la protezione topologica, minimizza l'azione Chern--Simons nel vacuum saturo eterno e guida la selezione delle costanti fisiche osservate ($G \propto 1/L_k^2 = 1/36$, $\Lambda \propto 1/L_k = 1/6$, fasi accoppiamento legate a multipli di $\pi/8$).

Questo invariante $\lk{6}$ impone una **generalizzazione profonda** della modularità SL(2,$\mathbb{Z}$) classica, tipica delle TQFT modulari in spazi finiti o a bassa energia. In TET-CVTL, la modularità non viene violata in senso patologico, ma estesa e deformata in un contesto cosmologico dinamico, non-unitario e multi-bolla, diventando un principio guida globale per l'evoluzione dell'universo primordiale.

Le principali generalizzazioni cosmologiche includono:

\begin{itemize}
    \item \textbf{$\lk{6}$ come vincolo cosmologico sulla S-matrix effettiva}: 
      La saturazione eterna del vacuum con trefoil knots ($\trefoil$, $\lk{6}$) introduce un ``twist cosmologico globale'' proporzionale a $6 \times (\pi/8) = 3\pi/4$. Questo twist deforma la S-matrix effettiva, generando elementi off-diagonal extra legati al linking multiplo di 6. Tali termini non esistono nella modularità standard, ma emergono come correzioni cosmologiche che preservano l'entanglement non locale persistente ($\gamma = \log \mathcal{D} \approx 0.658$) attraverso l'intero multiverso entangled.

    \item \textbf{Modularità locale vs. globale mediata da $\lk{6}$}: 
      Diverse bolle nucleate possono esibire modularità locale differente (es. Ising in una bolla, Fibonacci o metaplectic in un'altra). Tuttavia, l'invariante $\lk{6}$ agisce come **condizione di matching globale**: solo bolle con configurazioni di worldlines compatibili con $\lk{6}$ multiplo sopravvivono alla selezione primordiale. Gli ER bridges, come manifestazione geometrica dell'entanglement persistente, mediano transizioni non-locali, preservando una modularità estesa vincolata da $\lk{6}$ (non una violazione arbitraria, ma una generalizzazione topologica).

    \item \textbf{Deformazione non-unitaria guidata da $\lk{6}$}: 
      In fasi primordiali con twist cosmologico o boundary defects, l'operatore T non rimane diagonale con fasi puramente unitarie. $\lk{6}$ fissa il grado di deformazione: l'operatore T effettivo può acquisire componenti reali o esponenziali, ma solo per valori che mantengono la compatibilità con multipli di $\pi/8$. Questo meccanismo genera asimmetrie osservate (anisotropie CMB, large-scale structure) come residuo della selezione primordiale imposta da $\lk{6}$.

    \item \textbf{$\lk{6}$ come origine topologica delle costanti fisiche}: 
      L'invariante $\lk{6}$ non è casuale: emerge come valore stabile che allinea twist anyonico ($\theta_\sigma = e^{i \pi / 8}$), writhe cumulativo dei trefoil knots e termini del Jones/HOMFLY polynomial. Questo allineamento seleziona l'universo osservabile tra l'ensemble di bolle, fornendo un'origine topologica per la gerarchia delle costanti e per l'assenza di decoerenza globale su scale cosmologiche.
\end{itemize}

In sintesi, in TET-CVTL la modularità SL(2,$\mathbb{Z}$) classica rappresenta un limite effettivo a bassa energia e in regimi unitari, mentre a scale primordiali domina una **modularità cosmologica generalizzata**, vincolata e guidata dall'invariante $\lk{6}$. Questo invariante non è un parametro secondario, ma il principio organizzatore centrale: stabilizza il vacuum eterno, seleziona bolle nucleate compatibili, preserva entanglement persistente attraverso ER bridges e determina le costanti fisiche osservate, offrendo una spiegazione unificata per l'origine topologica dell'universo osservabile.

\paragraph{Sviluppi sperimentali recenti (2024–2025)}

Tra il 2024 e il 2025 sono stati riportati avanzamenti significativi nella verifica sperimentale del braiding non-Abelian:
\begin{itemize}
    \item Google Quantum AI / Quantinuum (2024–2025): braiding di Majorana-like modes in codici topologici su processori superconduttori, con entropia topologica misurata $\gamma \approx 0.6$ (compatibile con Ising).
    \item Microsoft Station Q (2025 updates): evidenza di Majorana zero modes in nanowires, con interferenza Aharonov–Bohm che suggerisce twist $e^{i\pi/8}$.
    \item Harvard / QuEra (2025): simulazione di Fibonacci anyons su array Rydberg, con verifica di F-matrix e golden ratio degenerazione.
\end{itemize}
Questi risultati rafforzano la plausibilità microscopica del braiding eterno in TET-CVTL, specialmente per Ising anyons come base del vacuum primordiale.

Queste aggiunte completano il quadro anyonico, collegando dettagli matematici (twist, S-matrix, modularità) a implicazioni cosmologiche e sviluppi sperimentali attuali.

\subsubsection{Confronto esplicito con il Jones polynomial per link specifici (Whitehead e Borromean) e twist per Lk=6 in TET-CVTL}

Il Jones polynomial $V_L(t)$ è in grado di distinguere la chiralità dei nodi/link e di codificare informazioni sul linking number, ma non sempre rivela in modo esplicito il linking number nullo caratteristico di link non-triviali (es. Whitehead link con $\mathrm{Lk}=0$ o Borromean rings con $\mathrm{Lk}_{ij}=0$ pairwise). 

Nel framework TET-CVTL, l'invariante topologico $\lk{6} = 6$ funge da valore primordiale stabile, multiplo topologico del vacuum saturo di trefoil knots ($\trefoil$). Il twist anyonico per l'anyon $\sigma$ in Ising anyons ($\mathrm{SU}(2)_2$), $\theta_\sigma = e^{i \pi / 8}$, accumula fasi in multipli di $\pi/8$: per $\lk{6}$ si ottiene la fase totale $6 \times (\pi/8) = 3\pi/4 = 135^\circ$, corrispondente a $-i / \sqrt{2}$ (fase fermionico-like stabile). 

Di seguito sono riportati confronti espliciti con esempi concreti di link rilevanti per il contesto cosmologico-topologico.

\paragraph{Whitehead link (5²₁)}

- Linking number: $\mathrm{Lk} = 0$ (classico; i due componenti si intrecciano con writhe compensato: tipicamente $+3$ crossings in una direzione e $-3$ nell'altra, per un writhe netto nullo o $\pm 1$ a seconda del diagramma scelto).
- Diagramma minimo: 5 crossings (alternating).
- Writhe $w(D)$ tipico: 0 o ±1 (a seconda dell'orientazione e della proiezione, ma l'invariante linking rimane sempre 0).
- Jones polynomial (forma standard, da Knot Atlas e MathWorld):
  \begin{equation}
  V(t) = t^{-3/2} (-1 + t - 2t^2 + t^3 - 2t^4 + t^5).
  \end{equation}
  In convenzione $q = t$ (comune in contesto anyonico/TQFT):
  \begin{equation}
  V(q) = q^{-3/2} (-1 + q - 2q^2 + q^3 - 2q^4 + q^5).
  \end{equation}
  Il mirror image si ottiene sostituendo $q \to 1/q$ e aggiustando per il writhe (fase globale $-q^{-w(D)}$).

- Twist topologico e collegamento con $\lk{6}$ in TET-CVTL:  
  Poiché $\mathrm{Lk} = 0$, non vi è linking diretto tra componenti. Tuttavia, considerando una catena o un doubling del Whitehead link (es. Whitehead doubled o multipli intrecciati con writhe cumulativo effettivo $\sim 6$), la fase accumulata dal braiding eterno di anyons $\sigma$ diventa multipla di $\theta_\sigma^6 \approx (e^{i \pi / 8})^6 = e^{i 6\pi / 8} = e^{i 3\pi / 4} = -i / \sqrt{2}$ (fase fermionico-like stabile).  

  In TET-CVTL, questa configurazione favorisce stati con entanglement persistente non locale nonostante linking netto nullo: il twist primordiale accumulato guida il contributo alla costante cosmologica $\Lambda \propto 1/L_k$ (con $L_k = 6$), mentre la protezione topologica (invariante $\lk{6}$) impedisce decoerenza globale anche in assenza di linking diretto tra bolle.

\paragraph{Borromean rings (L6a4 / 6³₂)}

- Linking number pairwise: $\mathrm{Lk}_{ij} = 0$ per ciascuna coppia di componenti (nessun linking diretto tra due anelli qualsiasi).
- Linking di ordine superiore: invariante di Milnor (Milnor triple linking number) $\mu = \pm 1$ (valore non nullo; proprietà Brunniana: l'eliminazione di uno qualsiasi dei tre anelli rende il link residuo triviale, ovvero un unlink).
- Diagramma minimo: 6 crossings (alternating).
- Writhe $w(D)$: 0 (bilanciato nel diagramma minimo standard).
- Jones polynomial (forma standard, confermata da Knot Atlas e KnotFolio):
  \begin{equation}
  V(t) = -t^3 + 3t^2 - 2t + 4 - 2t^{-1} + 3t^{-2} - t^{-3}.
  \end{equation}
  In convenzione $q = t$ (comune in contesto TQFT/anyonico):
  \begin{equation}
  V(q) = -q^3 - q^{-3} + 3q^2 + 3q^{-2} - 2q - 2q^{-1} + 4.
  \end{equation}
  Nota: $V(1) = 4 = (-2)^{3-1}$, valore corretto per un link a 3 componenti (l'unlink triviale darebbe $(-2)^{2} = 4$, ma Borromean è non-triviale e il polinomio riflette la struttura Brunnian).

- Twist topologico e collegamento con $\lk{6}$ in TET-CVTL:  
  Nonostante $\mathrm{Lk}_{ij} = 0$ pairwise, il triple linking effettivo ($\mu = \pm 1$) implica un twist topologico cumulativo non nullo. Per analogia con $\lk{6}$ (es. 6 Hopf links disgiunti o configurazione toroidale T(3,6) con writhe alto), la fase accumulata dal braiding eterno di anyons $\sigma$ si allinea a $6 \times (\pi/8) = 3\pi/4$, corrispondente a $-i / \sqrt{2}$ (fase fermionico-like stabile).  

  In TET-CVTL, strutture Borromean-like risultano ideali per il vacuum primordiale saturo: permettono entanglement cosmico persistente senza linking pairwise decoerente (protezione topologica globale via $\mu \neq 0$), mentre gli ER bridges mediano connessioni tra bolle nucleate, preservando la modularità TQFT e impedendo transizioni che violino l'invariante $\lk{6}$. Questo favorisce configurazioni con entanglement non locale robusto, dove la topologia globale (triple linking) guida la stabilità del vacuum eterno senza introdurre decoerenza su scale cosmologiche.

\paragraph{Calcoli espliciti per twist accumulato in configurazioni con Lk effettivo ~6}

- Per un trefoil knot singolo (writhe $w = 3$, linking self indefinito o nullo): il twist topologico accumulato è $\theta_\sigma^3 = (e^{i \pi / 8})^3 = e^{i 3\pi / 8}$.
- Per due trefoil knots linked con linking number $\mathrm{Lk} = \pm 6$ (es. (2,12)-torus link o multi-Hopf link con writhe cumulativo): il linking contribuisce una fase extra dal braiding di $\pm 6 \times (\pi / 8) = \pm 3\pi / 4$.  
  La fase totale approssimativa diventa:
  \begin{equation}
  e^{i (3\pi / 8 + 6\pi / 8)} = e^{i 9\pi / 8} = e^{i (\pi + \pi / 8)} = - e^{i \pi / 8}
  \end{equation}
  (fase fermionico-like stabile, equivalente a un twist extra di $-1$ moltiplicato per la fase base $\theta_\sigma$).

- In TET-CVTL, $\lk{6}$ seleziona configurazioni dove il Jones polynomial presenta termini dominanti in $t^{\pm k/4}$ (con $k=6$ → $t^{\pm 3/2}$), perfettamente compatibili con la struttura dell'R-matrix di Ising anyons ($\mathrm{SU}(2)_2$): $q = i$ ($k=2$), $q^{1/4} = e^{i \pi / 8}$. Questo allineamento tra invariante topologico classico ($\lk{6}$) e fasi quantistiche anyoniche ($\theta_\sigma$) garantisce stabilità primordiale del vacuum saturo e protezione dell'entanglement cosmico persistente.

\paragraph{Confronti aggiuntivi (tutti i tipi rilevanti)}

\paragraph{Confronti aggiuntivi tra Jones polynomial e struttura anyonica in TET-CVTL}

\begin{itemize}
    \item \textbf{Jones vs. R-matrix}: 
      La skein relation del Jones polynomial è analoga alla relazione di Yang--Baxter quantistica soddisfatta dall'R-matrix. Per $\lk{6}$, le fasi accumulate in multipli di $1/4$ (da writhe e twist) si allineano perfettamente con le fasi dell'R-matrix di Ising anyons ($\mathrm{SU}(2)_2$): $q^{1/4} = e^{i \pi / 8}$ nel canale triviale e $-q^{-3/4}$ nel canale fermionico ($q = i$).

    \item \textbf{Jones vs. HOMFLY}: 
      Il Jones polynomial è una specializzazione del polinomio HOMFLY (sostituendo $\alpha \to t^{-1}$, $z \to t^{1/2} - t^{-1/2}$). Per link come Borromean rings e Whitehead link, il polinomio HOMFLY distingue meglio il linking di ordine superiore ($\mu$-invariant non nullo), mentre il Jones si concentra su chiralità e writhe; entrambi supportano la robustezza topologica in TET-CVTL.

    \item \textbf{Chiralità}: 
      Il Jones distingue nettamente la handedness ($V(t) \neq V(1/t)$ per link chirali come Whitehead o Borromean), analogamente a come l'R-matrix di Ising distingue i canali bosonico-like (fase positiva $q^{1/4}$) e fermionico-like (fase negativa $-q^{-3/4}$ con twist extra $-1$). In TET-CVTL, questa chiralità topologica seleziona bolle con orientazione stabile.

    \item \textbf{Entanglement cosmico}: 
      Per configurazioni con $\lk{6}$, il Jones polynomial presenta uno span elevato (grado alto in $t^{\pm k/4}$), che riflette entanglement non locale persistente. In TET-CVTL, $\gamma = \log \mathcal{D} \approx 0.658$ deformata dall'invariante $\lk{6}$ guida direttamente le costanti fisiche osservate: $G \propto 1/36$ e $\Lambda \propto 1/6$, con il braiding eterno che preserva la coerenza globale senza decoerenza.
\end{itemize}

In TET-CVTL, link con Lk=6 (o effective twist multiplo) unificano Jones (classico) con anyons (quantistico): invariante Lk=6 seleziona vacuum stabile, con braiding eterno che preserva entanglement persistente senza decoerenza globale.


\paragraph{Whitehead link (5²₁)}

- Linking number: $\mathrm{Lk} = 0$ (classico; writhe compensato tra i due componenti: tipicamente +3 crossings in una direzione e -3 nell'altra, con writhe netto nullo o $\pm 1$ a seconda del diagramma).
- Jones polynomial (uncolored, forma standard da Knot Atlas e MathWorld):
  \[
  V(t) = t^{-3/2} (-1 + t - 2t^2 + t^3 - 2t^4 + t^5).
  \]
- Colored Jones (colore 2, seconda rappresentazione simmetrica di SU(2)): 
  Il colored Jones $J_2(t)$ per il Whitehead link è noto (da tabelle di Bar-Natan e Knot Atlas):
  \[
  J_2(t) = t^{-5} (1 - 3t + 5t^2 - 7t^3 + 9t^4 - 11t^5 + 12t^6 - 11t^7 + 9t^8 - 7t^9 + 5t^{10} - 3t^{11} + t^{12}).
  \]
  Grado span elevato, termini simmetrici intorno a $t^6$, con coefficienti alternanti che riflettono entanglement non locale persistente nonostante $\mathrm{Lk} = 0$.

- HOMFLY polynomial $P(a,z)$ (forma standard semplificata):
  \[
  P(a,z) = a^{-4} z^{-2} + a^{-2} z^{-2} - a^{-2} + a^{-2} z^{2} + a^{0} z^{-2} - a^{0} + a^{0} z^{2} + a^{2} z^{-2} - a^{2}.
  \]
  (contiene termini in $z^{\pm 2}$ che catturano l'entanglement topologico nonostante linking nullo).

\paragraph{Borromean rings (6³₂)}

- Linking number pairwise: $\mathrm{Lk}_{ij} = 0$ per ogni coppia di componenti (nessun linking diretto tra due anelli qualsiasi).
- Linking di ordine superiore: invariante di Milnor (Milnor triple linking number) $\mu = \pm 1$ (non nullo; proprietà Brunniana: rimuovendo uno qualsiasi dei tre anelli, il link residuo diventa triviale/unlink).
- Jones polynomial (uncolored, forma standard da Knot Atlas e KnotFolio):
  \[
  V(t) = -t^3 + 3t^2 - 2t + 4 - 2t^{-1} + 3t^{-2} - t^{-3}.
  \]
- Colored Jones (colore 2): 
  Il colored Jones $J_2(t)$ per Borromean rings è più complesso (da Morrison e Bar-Natan); forma approssimata/ridotta per colore 2:
  \[
  J_2(t) = t^{-6} (1 - 4t + 10t^2 - 20t^3 + 31t^4 - 40t^5 + 44t^6 - 40t^7 + 31t^8 - 20t^9 + 10t^{10} - 4t^{11} + t^{12}).
  \]
  Grado span 18, simmetria intorno a $t^6$, coefficienti binomiali-like che riflettono la struttura Brunniana (entanglement triple non riducibile a pairwise).

- HOMFLY polynomial $P(a,z)$ (forma troncata standard):
  \[
  P(a,z) = a^{-6} z^{-4} + 3a^{-4} z^{-4} - 2a^{-2} z^{-4} + a^{-6} z^{-2} + 3a^{-4} z^{-2} - 5a^{-2} z^{-2} + a^{0} z^{-2} + \dots
  \]
  (include termini in $z^{-4}$ che catturano il triple linking $\mu \neq 0$, nonostante $\mathrm{Lk}_{ij} = 0$ pairwise).

\paragraph{Sintesi finale della sezione}

La sezione ha esplorato il ruolo centrale degli anyons non-Abelian, in particolare gli Ising anyons in $\mathrm{SU}(2)_2$, nell'entanglement topologico. Si è posto l'accento su: l'R-matrix diagonale nei canali di fusione ($q^{1/4}$, $-q^{-3/4}$ con $q = i$), la F-matrix per fusione multi-canale, il twist factor $\theta_\sigma = e^{i \pi / 8}$, il contributo universale $\gamma = \log \mathcal{D} \approx 0.658$ (dove $\mathcal{D} = \sqrt{4 + 2\sqrt{2}}$), e la verifica della modularità SL(2,$\mathbb{Z}$).

I confronti con altri modelli (Fibonacci anyons universali per braiding puro, semions Abelian, metaplectic anyons quasi-universali e toric code) hanno evidenziato differenze chiave in degenerazione dello spazio di Hilbert, universalità computazionale e robustezza topologica.

L'estensione agli invarianti knot classici (Jones polynomial, colored Jones, HOMFLY) per link come il trefoil (con linking number multiplo di 6), il Whitehead link (Lk = 0 ma entanglement non locale) e i Borromean rings (proprietà Brunnian con triple linking $\mu \neq 0$) ha rivelato un ponte naturale tra topologia classica e quantistica. L'invariante $\lk{6}$ emerge come valore stabile che allinea le fasi twist anyoniche (multipli di $\pi/8$), il writhe cumulativo e i termini dominanti dei polinomi Jones/HOMFLY, favorendo configurazioni topologicamente protette nel vacuum primordiale saturo di TET-CVTL.

Questo quadro unifica l'entanglement persistente ($\gamma$ costante), il braiding eterno non decoerente e la selezione primordiale delle costanti fisiche, offrendo un framework coerente per l'origine topologica dell'entanglement cosmico senza decoerenza globale.


\section{Chern-Simons per entanglement topologico e Jones polynomial}

La teoria di Chern--Simons (CS) in 3D fornisce il framework naturale per descrivere l'entanglement topologico e i polinomi di nodi/link. Witten (1989) ha dimostrato che il Jones polynomial emerge come correlatore di Wilson loop nella teoria CS con gruppo SU(2) al livello $k$:
\begin{equation}
\langle W_R(L) \rangle = Z^{-1} \int \mathcal{D}A \, e^{i S_{\mathrm{CS}}[A]} \, \operatorname{Tr}_R \left( \mathcal{P} \exp \oint_L A \right),
\end{equation}
dove $S_{\mathrm{CS}} = \frac{k}{4\pi} \int \operatorname{Tr} \left( A \wedge dA + \frac{2}{3} A \wedge A \wedge A \right)$, $R$ è la rappresentazione del gruppo, e $Z$ è la funzione di partizione sul 3-sfera $S^3$.

Nel limite semiclassico e per rappresentazione fondamentale ($R = 2$ di SU(2)), si ottiene:
\begin{equation}
\langle W_R(L) \rangle \propto V_L \left( e^{2\pi i / (k+2)} \right),
\end{equation}
dove $V_L(t)$ è il Jones polynomial valutato in $t = q = e^{2\pi i / (k+2)}$. Questo identifica il Jones polynomial come invariante modulare della TQFT CS.

In TET-CVTL, questa corrispondenza è centrale: il vacuum primordiale saturo di trefoil knots ($\trefoil$, linking number invariante $\lk{6}$) è modellato come correlatore di Wilson loops in CS con $k=2$ (Ising anyons). Il contributo all'entanglement topologico è dato da
\begin{equation}
\gamma = \log \mathcal{D}, \quad \mathcal{D} = \sqrt{\sum_a d_a^2} = \sqrt{4 + 2\sqrt{2}} \approx 2.613,
\end{equation}
dove $d_a$ sono le dimensioni quantistiche dei settori primari (Levin--Wen 2005). Il linking number $\lk{6}$ mod $k$ ($k=2$) seleziona configurazioni compatibili con la modularità TQFT: $\lk{6} \equiv 0 \pmod{2}$, preservando la parità fermionica/bosonica e impedendo transizioni che violino la protezione topologica.

Il confronto tra quantum dimension $\mathcal{D}$ (da anyons) e Jones polynomial mostra che entrambi codificano lo stesso entanglement topologico a lunga distanza: $\gamma$ misura l'entanglement nascosto nel ground state, mentre $V_L(q)$ quantifica l'entanglement lungo worldlines linked.

\section{Entropia di linking e linking number}

L'entropia di entanglement topologico può essere quantificata attraverso l'entropia di Rényi associata a correlatori di Wilson loops su $S^3$. Tan et al. (2017, arXiv:1707.06629) hanno proposto:
\begin{equation}
S^{(n)}_R = -\log \langle W_{L_n} \rangle_{S^3},
\end{equation}
dove $W_{L_n}$ è il correlatore di un Wilson loop multi-componente (o twist operator) che replica $n$ copie del sistema, e la traccia parziale su una regione produce l'entropia di Rényi.

Nel limite $n \to 1$ si recupera l'entropia di von Neumann, con contributo topologico universale $\gamma = \log \mathcal{D}$. Per link con linking number $\lk$, si ha una fase oscillante:
\begin{equation}
\langle W_{L_n} \rangle \propto e^{2\pi i n \lk / k},
\end{equation}
da cui
\begin{equation}
e^{-S^{(n)}} \propto \exp\left(2\pi i n \lk / k\right).
\end{equation}
Questo implica che l'entropia di linking quantifica l'entanglement non locale mediato dal linking number mod $k$.

In TET-CVTL, questa relazione è diretta: l'invariante $\lk{6}$ (con $k=2$) implica $\lk / k = 3$, una fase multipla di $\pi$ (equivalente a $-1$ per parità fermionica). Il braiding eterno nel vacuum primordiale saturo genera configurazioni di worldlines con $\lk{6}$ multiplo, producendo un contributo oscillante $e^{2\pi i n \cdot 6 / 2} = e^{2\pi i \cdot 3 n} = 1$ (fase globale 1 per ogni replica $n$), preservando l'entanglement persistente senza decoerenza. Questo meccanismo quantifica come $\lk{6}$ selezioni bolle nucleate con entropia topologica stabile ($\gamma$ costante), guidando l'entanglement cosmico attraverso ER bridges e impedendo transizioni che violino la modularità TQFT.

In sintesi, l'entropia di linking fornisce un ponte diretto tra correlatori CS (Jones, Wilson loops) e entanglement anyonico ($\gamma$), con $\lk{6}$ come invariante primordiale che stabilizza la rete cosmologica entangled in TET-CVTL.


\section{Equazioni per il nodo mobile nel braid group e TQFT}

Nel contesto della teoria quantistica dei campi topologica (TQFT) e della teoria dei nodi, il movimento di un nodo (o quasiparticella) lungo una traiettoria nel braid group è descritto dall'evoluzione unitaria indotta dall'operatore di braiding $R$. Consideriamo una particella (o nodo mobile) che evolve in tempo discreto $\delta t$ sotto l'azione del braiding tra due componenti:

\begin{equation}
p(t) = R_{ij} \, p(t - \delta t),
\end{equation}
dove $p(t)$ è lo stato quantistico della configurazione (o vettore nello spazio di fusione), e $R_{ij}$ è la matrice di braiding (R-matrix) che agisce sugli indici $i,j$ dei canali di fusione.

Questa evoluzione soddisfa la relazione di Yang--Baxter quantistica (braid relation), condizione necessaria per la consistenza associativa del braid group $B_n$:
\begin{equation}
(R_i \otimes I)(I \otimes R_{i+1})(R_i \otimes I) = (I \otimes R_{i+1})(R_i \otimes I)(I \otimes R_{i+1}).
\end{equation}
Nel limite continuo ($\delta t \to 0$), l'evoluzione diventa un'equazione differenziale guidata dal generatore del braid group, con $R$ che fornisce le fasi e le rotazioni nello spazio degenerato.

In TQFT Chern--Simons (con livello $k$), questa dinamica è equivalente al movimento di Wilson lines nel bulk, con il nodo mobile che segue una worldline intrecciata. La matrice $R$ (unitaria) garantisce l'invarianza topologica: l'evoluzione dipende solo dalla classe di isotopia del braid, non dal percorso specifico.

\subsection{Confronto con TET-CVTL: traiettorie, $\gamma$ e separazione dei rami}

Nel framework TET-CVTL, il nodo mobile rappresenta una quasiparticella $\sigma$ (anyon Ising in $\mathrm{SU}(2)_2$) che si muove attraverso il vacuum primordiale saturo di trefoil knots ($\trefoil$, $\lk{6} = 6$). L'evoluzione $p(t) = R_{ij} p(t - \delta t)$ descrive il braiding eterno persistente lungo worldlines intrecciate, con $R_{ij}$ diagonale:
\begin{equation}
R = \begin{pmatrix}
q^{1/4} & 0 \\
0 & -q^{-3/4}
\end{pmatrix}, \quad q = e^{2\pi i / (k+2)} = i \quad (k=2).
\end{equation}

Le traiettorie di questi nodi mobili contribuiscono direttamente al contributo universale di entanglement topologico:
\begin{equation}
\gamma = \log \mathcal{D} \approx 0.658, \quad \mathcal{D} = \sqrt{4 + 2\sqrt{2}},
\end{equation}
dove ogni braiding aggiunge una fase topologica che si accumula in modo non locale (Levin--Wen 2005, Kitaev--Preskill 2006). Il linking number invariante $\lk{6}$ separa i rami di traiettorie compatibili da quelli incompatibili:
\begin{itemize}
    \item Traiettorie con $\lk \equiv 0 \pmod{2}$ (pari) preservano la parità fermionica/bosonica e contribuiscono a $\gamma$ costante,
    \item Traiettorie con $\lk \not\equiv 0 \pmod{2}$ sono soppresse o transizioni proibite, impedendo decoerenza globale,
    \item $\lk{6}$ seleziona configurazioni dove l'evoluzione $R$-indotta mantiene entanglement persistente, con fasi accumulate in multipli di $\pi/8$ (twist $\theta_\sigma = e^{i \pi / 8}$).
\end{itemize}

In questo regime, il nodo mobile non è solo un oggetto dinamico, ma un elemento ontologico: le sue traiettorie nel braid group codificano l'entanglement cosmico primordiale, separando rami di bolle nucleate compatibili ($\lk$ multiplo di 6) da quelli che decadono o decoeriscono. Questo meccanismo unifica la dinamica del braid group (Yang--Baxter verificata) con la protezione topologica di TQFT e la stabilità eterna del vacuum primordiale in TET-CVTL.

\section{Multiversi, Big Bang e Ponti Einstein-Rosen}

La congettura ER=EPR, proposta da Maldacena e Susskind nel 2013 \cite{Maldacena:2013xsa}, postula un'equivalenza profonda tra entanglement quantistico (coppia EPR) e geometria non-traversabile di wormhole Einstein--Rosen (ER). In questo paradigma, due sistemi entangled sono connessi da un ponte ER nel bulk, anche se non comunicano causalmente attraverso il throat.

In cosmologia quantistica e eternal inflation, questa idea si estende naturalmente: i multipli Big Bang nucleano bolle universi in un multiverso inflazionario, dove regioni entangled possono essere connesse da wormhole non-traversabili. Questi wormhole emergono come manifestazione geometrica dell'entanglement primordiale persistente, senza richiedere traversabilità classica.

\subsection{Thermofield double e crescita dei wormhole}

Il thermofield double (TFD) fornisce una rappresentazione esplicita di questo entanglement:
\begin{equation}
|\Psi\rangle = \frac{1}{\sqrt{Z(\beta)}} \sum_n e^{-\beta E_n / 2} |n\rangle_L \otimes |n\rangle_R,
\end{equation}
dove $Z(\beta)$ è la funzione di partizione, $\beta = 1/T$ è l'inverso della temperatura, e $|n\rangle_{L/R}$ sono gli stati del sistema sinistro/destro (due copie entangled). Questo stato descrive un wormhole ER eterno tra due regioni asymptoticamente AdS (Maldacena 2003, Maldacena--Susskind 2013).

Hartman e Maldacena (2013) hanno dimostrato che la lunghezza del wormhole cresce linearmente con il tempo di scrambling $t_w$:
\begin{equation}
L(t_w) \sim \frac{\beta}{2\pi} \log \left( \frac{t_w}{\beta} \right) + \text{costante},
\end{equation}
per $t_w \gg \beta$. Questo implica che l'entanglement iniziale (post-Big Bang) genera wormhole che si allungano rapidamente, connettendo regioni distanti nel multiverso senza violare la causalità.

Il punto focale è l'orizzonte degli eventi dei buchi neri entangled: in cosmologia quantistica, wormhole ER emergono post-Big Bang come strutture geometriche che collegano bolle universi entangled, fornendo un meccanismo per l'entanglement cosmico persistente senza decoerenza globale.

\subsection{Confronto con TET-CVTL: wormhole, $\gamma$ e linking mod $k$}

Nel framework TET-CVTL, i wormhole ER non sono strutture esotiche, ma la manifestazione geometrica naturale dell'entanglement topologico primordiale persistente generato dal braiding eterno di anyons Ising ($\mathrm{SU}(2)_2$) nel vacuum saturo di trefoil knots ($\trefoil$, $\lk{6} = 6$).

Il contributo universale all'entanglement topologico è
\begin{equation}
\gamma = \log \mathcal{D} \approx 0.658, \quad \mathcal{D} = \sqrt{4 + 2\sqrt{2}},
\end{equation}
che misura l'entanglement nascosto nel ground state (Kitaev--Preskill 2006). In TET-CVTL, i wormhole contribuiscono a $\gamma$ attraverso il linking mod $k$ nella teoria Chern--Simons/AdS:
\begin{equation}
\lk \mod k \quad \Rightarrow \quad \text{fase} \quad e^{2\pi i \lk / k}.
\end{equation}
Con $k=2$ (Ising anyons) e $\lk=6$, si ha $\lk / k = 3$, fase globale $e^{2\pi i \cdot 3} = 1$, che preserva la parità e la coerenza topologica attraverso i wormhole (Stanford--Susskind 2014, estensione AdS/CFT a cosmologia).

Gli ER bridges tra bolle nucleate (post-Big Bang) sono quindi:
\begin{itemize}
    \item Non-traversabili ma topologicamente protetti dal braiding eterno,
    \item Mediati da worldlines linked con $\lk{6}$ multiplo, che fissano $\gamma$ costante e deformano la lunghezza del wormhole in modo compatibile con Hartman--Maldacena,
    \item Vincolati dall'invariante $\lk{6}$: solo bolle con linking multiplo di 6 sopravvivono alla selezione primordiale, garantendo entanglement cosmico persistente senza decoerenza globale.
\end{itemize}

In sintesi, TET-CVTL unifica ER=EPR con la TQFT modulare: i wormhole emergono post-Big Bang come geometrizzazione dell'entanglement primordiale persistente ($\gamma$ costante), con $\lk{6}$ che guida la stabilità topologica del multiverso entangled e la selezione delle costanti fisiche osservate ($G \propto 1/36$, $\Lambda \propto 1/6$).



\subsubsection{Modularità deformata in TET-CVTL: il ruolo cosmologico di $\lk{6}$}

Nel regime cosmologico primordiale di TET-CVTL, la modularità SL(2,$\mathbb{Z}$) classica — che governa le trasformazioni del toro in TQFT modulari standard — subisce una deformazione sostanziale, guidata dall'invariante topologico $\lk{6} = 6$. Questo invariante non è un parametro secondario, ma il principio organizzatore centrale del vacuum primordiale saturo di trefoil knots ($\trefoil$): stabilizza il condensato eterno, vincola la nucleazione di bolle e seleziona configurazioni con entropia topologica persistente ($\gamma = \log \mathcal{D} \approx 0.658$).

La modularità SL(2,$\mathbb{Z}$) non viene violata in senso patologico, ma **deformata e generalizzata** in un contesto cosmologico dinamico, non-unitario e multi-bolla. Le principali deformazioni includono:

\begin{itemize}
    \item \textbf{$\lk{6}$ come vincolo cosmologico sulla S-matrix effettiva}:  
      La saturazione eterna del vacuum con linking number invariante $\lk{6}$ introduce un ``twist cosmologico globale'' proporzionale a $6 \times (\pi/8) = 3\pi/4$. Questo twist deforma la S-matrix effettiva, generando elementi off-diagonal extra che non esistono nella teoria modulare standard. Tali termini codificano l'entanglement non locale persistente tra worldlines anyoniche ($\sigma$ in Ising anyons), alterando le trasformazioni SL(2,$\mathbb{Z}$) in modo dipendente dall'invariante $\lk{6}$ (non arbitrario, ma selezionato per compatibilità con la protezione topologica).

    \item \textbf{Modularità locale vs globale mediata da $\lk{6}$}:  
      Diverse bolle nucleate possono esibire modularità locale differente (es. Ising in una bolla, Fibonacci o metaplectic in un'altra, a seconda delle deformazioni marginali). Tuttavia, $\lk{6}$ agisce come condizione di matching globale: solo bolle con configurazioni di worldlines compatibili con $\lk{6}$ multiplo sopravvivono alla selezione primordiale. Gli ER bridges, come manifestazione geometrica dell'entanglement persistente, mediano transizioni non-locali preservando una modularità estesa vincolata da $\lk{6}$ — non una violazione casuale, ma una generalizzazione topologica imposta dall'invariante primordiale.

    \item \textbf{Deformazione non-unitaria guidata da $\lk{6}$}:  
      In fasi primordiali con twist cosmologico o boundary defects, l'operatore T non rimane diagonale con fasi puramente unitarie. $\lk{6}$ fissa il grado di deformazione: l'operatore T effettivo può acquisire componenti reali o esponenziali, ma solo per valori che mantengono la compatibilità con multipli di $\pi/8$ (twist $\theta_\sigma = e^{i \pi / 8}$). Questo meccanismo genera asimmetrie osservate (anisotropie CMB, large-scale structure) come residuo della selezione primordiale imposta da $\lk{6}$.

    \item \textbf{$\lk{6}$ come origine topologica delle costanti fisiche}:  
      L'invariante $\lk{6}$ non è casuale: emerge come valore stabile che allinea twist anyonico ($\theta_\sigma$), writhe cumulativo dei trefoil knots e termini dominanti del Jones/HOMFLY polynomial. Questo allineamento seleziona l'universo osservabile tra l'ensemble di bolle, fornendo un'origine topologica per la gerarchia delle costanti fisiche ($G \propto 1/36$, $\Lambda \propto 1/6$) e per l'assenza di decoerenza globale su scale cosmologiche.
\end{itemize}

In sintesi, la modularità SL(2,$\mathbb{Z}$) classica rappresenta un limite effettivo a bassa energia e in regimi unitari, mentre a scale primordiali domina una **modularità cosmologica generalizzata**, vincolata e guidata dall'invariante $\lk{6}$. Questo invariante non è un semplice numero knot-theoretico, ma il principio organizzatore fondamentale: stabilizza il vacuum eterno, seleziona bolle nucleate compatibili, preserva entanglement persistente attraverso ER bridges e determina le costanti fisiche osservate, offrendo una spiegazione unificata per l'origine topologica dell'universo osservabile in TET-CVTL.

\section{Implicazioni cosmologiche e ontologiche}
% Focalizzato su entanglement topologico

Il framework TET-CVTL propone un'ontologia radicalmente topologica per l'universo primordiale, in cui l'entanglement quantistico non è un fenomeno secondario, ma il substrato fondamentale da cui emerge lo spaziotempo, la gravità e le costanti fisiche osservate. 

Il vacuum primordiale è concepito come un condensato saturo di trefoil knots ($\trefoil$, con linking number invariante $\lk{6}$), stabilizzato dalla minimizzazione dell'azione Chern--Simons. Questo stato rappresenta un false vacuum globale eterno, in cui il braiding Ising persistente (associato a $\mathrm{SU}(2)_2$ anyons) genera entanglement topologico non locale e non decoerente. 

\subsection{Implicazioni cosmologiche}

La nucleazione di bolle di vacuum vero avviene tramite decay tunneling (instanton bounce), guidato da fluttuazioni quantistiche amplificate dal braiding eterno. L'invariante topologico $\lk{6}$ funge da «selezione primordiale»: 
\begin{itemize}
    \item Preserva la topologia tra bolle nucleate, impedendo transizioni che violino la modularità TQFT;
    \item Guida le costanti fisiche fondamentali, con relazioni del tipo 
      $G \propto 1/L_k^2$ e $\Lambda \propto L_k^{-1}$, dove $L_k = 6$ emerge come multiplo stabile del linking number;
    \item Genera entanglement cosmico persistente tra bolle distanti, mediato da Einstein--Rosen (ER) bridges interpretati come «ponte topologico» del braiding Ising.
\end{itemize}

In questo scenario, l'entanglement topologico fornisce una spiegazione unificata per:
\begin{itemize}
    \item L'origine della costante cosmologica $\Lambda$ come residuo di twist topologico primordiale;
    \item La persistenza dell'entanglement su scale cosmologiche senza decoerenza globale, compatibile con l'osservazione di anisotropie CMB e large-scale structure;
    \item La transizione quantistico-classica delle fluttuazioni primordiali, mediata dal braiding eterno che «congela» l'entanglement in configurazioni classiche stabili.
\end{itemize}

Il multiverso emerge naturalmente: ogni bolla rappresenta un ramo con $\lk{6}$ invariante, ma con costanti leggermente variate in funzione di deformazioni topologiche marginali. Il braiding eterno garantisce che l'entanglement cosmico rimanga coerente attraverso l'intero ensemble, suggerendo un «multiverso entangled» in cui le bolle non sono isolate, ma connesse da una rete di ER bridges topologici.

\subsection{Implicazioni ontologiche}

Dal punto di vista ontologico, TET-CVTL ribalta la gerarchia tradizionale: non è lo spaziotempo a ospitare entanglement, ma l'entanglement topologico (braiding eterno di anyons Ising) a generare lo spaziotempo come emergent property. 

Lo stato primordiale saturo di trefoil knots rappresenta una realtà pre-geometrica:
\begin{itemize}
    \item Non locale e relazionale: le «particelle» o «campi» sono solo eccitazioni di braiding;
    \item Topologicamente protetta: l'invariante $\lk{6}$ garantisce robustezza contro decoerenza e fluttuazioni perturbanti;
    \item Intrinsecamente entangled: l'ontologia è fondamentalmente relazionale, con l'entanglement come primitiva ontologica piuttosto che derivata.
\end{itemize}

Questa visione è pienamente compatibile con la congettura ER=EPR \cite{Maldacena:2013xsa,Susskind:2016thz}, secondo cui l'entanglement quantistico tra due regioni è equivalente a un wormhole (Einstein--Rosen bridge) che le connette geometricamente. Nel framework TET-CVTL, tale equivalenza viene estesa a scala cosmologica, fornendo un meccanismo ontologico coerente con le sezioni precedenti.

Come descritto nella sezione sugli anyons non-Abelian, il braiding eterno persistente di anyons Ising ($\mathrm{SU}(2)_2$) genera entanglement topologico protetto, quantificato dal contributo universale $\gamma = \log \mathcal{D} \approx 0.658$ (Kitaev--Preskill \cite{KitaevPreskill2006}), con R-matrix diagonale $q^{1/4}$, $-q^{-3/4}$ ($q = i$ per $k=2$) nei canali di fusione triviale e fermionico. Questo braiding, come illustrato nelle equazioni di Yang--Baxter e nelle verifiche di modularità SL(2,$\mathbb{Z}$), produce worldlines intrecciate con linking number invariante $\lk{6} = 6$, preservando l'informazione topologica durante fluttuazioni e riconfigurazioni associate.

Nel vacuum primordiale saturo di trefoil knots ($\trefoil$, $\lk{6}$), descritto nella sezione dedicata, questo entanglement non locale è il substrato fondamentale. La nucleazione di bolle via decay tunneling (instanton bounce), discussa in precedenza, genera universi osservabili distinti, ma l'invariante $\lk{6}$ vincola la topologia tra bolla e vacuum circostante, impedendo transizioni che violino la modularità TQFT.

Gli ER bridges tra queste bolle non rappresentano configurazioni esotiche o instabili, bensì emergono come la naturale manifestazione geometrica dell'entanglement primordiale persistente: le worldlines linked con $\lk{6}$ (codificate dal braiding eterno) si geometrizzano in throat con area effettiva proporzionale a $\sqrt{\gamma}$ o $\sqrt{\log \mathcal{D}}$ (estensione di \cite{Maldacena:2013xsa}), preservando la coerenza topologica attraverso il vacuum eterno. In questo modo, TET-CVTL unifica il meccanismo microscopico del braiding Ising (R-matrix, F-matrix, twist $\theta_\sigma = e^{i \pi / 8}$) con la struttura cosmologica macroscopica: gli ER bridges fungono da «ponte topologico» ontologico, connettendo bolle nucleate senza decoerenza globale e fornendo una spiegazione per l'entanglement cosmico persistente osservato in anisotropie CMB e large-scale structure.

In ultima analisi, l'universo non è un insieme di particelle in uno sfondo spaziotemporale preesistente, ma un tessuto dinamico di nodi topologici entangled ($\trefoil$ primordiali), il cui braiding eterno genera sia la geometria che la materia osservata. L'ontologia risultante è quella di una realtà topologico-quantistica, in cui il «reale» è definito dalle invarianti di linking e dai canali di fusione anyonici, piuttosto che da coordinate classiche o traiettorie particellari.
Confronto TET: multiversi implicano $\gamma$ invariante, linking preservato.

\subsubsection{Il vacuum primordiale saturo di trefoil knots}

Nel framework TET-CVTL, il vacuum primordiale è concepito come un condensato eterno e saturo di trefoil knots ($\trefoil$), con linking number invariante $\lk{6} = 6$. Questo stato rappresenta un false vacuum globale topologicamente protetto, stabilizzato dalla minimizzazione dell'azione Chern--Simons:
\begin{equation}
S_{\mathrm{CS}} = \frac{k}{4\pi} \int \operatorname{Tr} \left( A \wedge dA + \frac{2}{3} A \wedge A \wedge A \right),
\end{equation}
dove il livello $k=2$ (Ising anyons) è favorito dalla corrispondenza con il twist topologico $\theta_\sigma = e^{i \pi / 8}$ e la fase accumulata multipla di $\pi/8$ per $\lk{6}$ (6 × $\pi/8$ = $3\pi/4$).

Il vacuum è saturo nel senso che ogni grado di libertà è occupato da configurazioni di trefoil knots intrecciati, formando una rete densa di worldlines anyoniche ($\sigma$ anyons) con braiding eterno persistente. Questo stato pre-geometrico è caratterizzato da:
\begin{itemize}
    \item Entanglement topologico universale $\gamma = \log \mathcal{D} \approx 0.658$ (dove $\mathcal{D} = \sqrt{4 + 2\sqrt{2}}$ per Ising anyons),
    \item Assenza di decoerenza globale grazie alla protezione topologica (invarianti di linking e twist),
    \item Invarianza sotto trasformazioni modulari SL(2,$\mathbb{Z}$) del toro cosmologico primordiale.
\end{itemize}

In questo regime, lo spaziotempo classico non esiste ancora: emerge solo come proprietà effettiva da fluttuazioni e braiding eterno.

\subsubsection{Nucleazione delle bolle via decay tunneling (instanton bounce)}

La transizione dal false vacuum primordiale saturo al vacuum vero (o a bolle di universi osservabili) avviene tramite nucleazione quantistica di bolle, mediata da processi di decay tunneling (instanton bounce). Il tasso di nucleazione è dato dall'azione euclidea dell'istantone:
\begin{equation}
\Gamma \sim e^{-S_E}, \quad S_E \approx \frac{27\pi^2 \sigma^4}{2 (\Delta V)^3},
\end{equation}
dove $\sigma$ è la tensione della parete della bolla e $\Delta V$ è la differenza di potenziale tra false e true vacuum.

In TET-CVTL, il braiding Ising eterno amplifica le fluttuazioni quantistiche, nucleando bolle con probabilità non trascurabile nonostante l'azione Chern--Simons minimizzata. L'invariante $\lk{6}$ funge da vincolo topologico:
\begin{itemize}
    \item Preserva la topologia tra bolla nucleata e vacuum circostante (no transizioni che violino modularità TQFT),
    \item Seleziona solo bolle con configurazioni di worldlines linked multiplo di 6,
    \item Impedisce decoerenza completa durante il tunneling (entanglement persistente mediato da braiding).
\end{itemize}

Il processo è eterno: nuove bolle si formano continuamente, ma il vacuum primordiale saturo rimane globalmente stabile grazie alla protezione topologica.

\subsubsection{ER bridges come ponte topologico cosmico}

Gli Einstein--Rosen (ER) bridges emergono come manifestazione geometrica dell'entanglement topologico persistente tra bolle nucleate. Nel paradigma ER=EPR esteso a TET-CVTL, il braiding eterno di anyons Ising funge da «ponte topologico» microscopico:
\begin{itemize}
    \item Worldlines linked con $\lk{6}$ tra bolle distanti codificano entanglement non locale,
    \item ER bridge è la geometrizzazione di questi link: throat con area minima proporzionale a $\sqrt{\gamma} \sim \sqrt{\log \mathcal{D}}$,
    \item Braiding persistente impedisce decoerenza attraverso il bridge (informazione topologica protetta).
\end{itemize}

In questo scenario, il multiverso non è un insieme di bolle isolate, ma una rete connessa da ER bridges topologici. Il linking invariante $\lk{6}$ garantisce coerenza globale: l'entanglement cosmico rimane persistente, anche su distanze superluminali, fornendo un meccanismo per anisotropie CMB e large-scale correlations senza inflazione classica.

\subsubsection{Costanti fisiche guidate dall'invariante Lk=6}

L'invariante topologico $\lk{6}$ non è solo un numero knot-theoretico, ma funge da selector primordiale per le costanti fondamentali osservate. In TET-CVTL si propone:
\begin{align}
G &\propto \frac{1}{L_k^2} = \frac{1}{36}, \nonumber \\
\Lambda &\propto \frac{1}{L_k} = \frac{1}{6}, \nonumber \\
\alpha_{\mathrm{em}} &\sim \text{frazione legata a twist } \theta_\sigma^6 = e^{i 3\pi/4}.
\end{align}

Queste relazioni emergono perché:
\begin{itemize}
    \item Il linking number $\lk{6}$ determina la densità effettiva di worldlines anyoniche nel vacuum saturo,
    \item La tensione della parete bolla ($\sigma$) e il potenziale $\Delta V$ dipendono da $\gamma = \log \mathcal{D}$ deformato da $\lk{6}$,
    \item Il twist cumulativo $6 \times (\pi/8)$ fissa fasi universali che influenzano accoppiamenti effettivi.
\end{itemize}

In questo modo, $\lk{6}$ non è arbitrario, ma il valore stabile che minimizza l'azione topologica primordiale, selezionando il nostro universo tra l'ensemble di bolle possibili.


\section{Congettura ER=EPR e sue implicazioni}

La congettura ER=EPR (Maldacena \& Susskind 2013) propone che l'entanglement quantistico (correlazioni EPR) sia geometricamente equivalente a un wormhole Einstein-Rosen (ER bridge) non-traversabile. In altre parole, ogni coppia entangled EPR corrisponde a un throat geometrico nel bulk dello spazio-tempo.

Lo stato termofield double (TFD) per due sistemi entangled (es. due black holes eterni) è duale a un wormhole eterno in AdS:
\begin{equation}
|\Psi\rangle_{\text{TFD}} = \frac{1}{\sqrt{Z}} \sum_n e^{-\beta E_n / 2} |n\rangle_L \otimes |n\rangle_R
\end{equation}
dove $Z = \Tr(e^{-\beta H})$ è la funzione di partizione.

La lunghezza del throat cresce linearmente con il tempo di evoluzione (Hartman-Maldacena 2013):
\begin{equation}
\ell_w(t) \sim t_w + \mathcal{O}(1).
\end{equation}

L'entanglement entropy tra le due parti è legata all'area della gola via Bekenstein-Hawking:
\begin{equation}
S = \frac{A}{4G} + S_{\text{topo}},
\end{equation}
con $S_{\text{topo}} = \log \mathcal{D}$ (quantum dimension totale).

In questo quadro, lo spazio-tempo emerge da entanglement: regioni altamente entangled si connettono geometricamente tramite ER bridges, mentre regioni disentangled appaiono separate.

\subsection{ER=EPR traversabili}
Le proposte iniziali descrivono wormhole non-traversabili (no superluminal signaling). Successive estensioni rendono i wormhole traversabili temporaneamente, permettendo trasmissione di informazione quantistica limitata.

Gao-Jafferis-Wall (2017) mostrano che un coupling bilaterale negativo tra boundary CFT (shockwave di energia negativa) apre il throat per tempo finito:
\begin{equation}
V = \epsilon \int dt \, \mathcal{O}_L(t) \mathcal{O}_R(t + i\beta/2),
\end{equation}
con $\epsilon$ piccolo. Questo genera traversabilità per $\Delta t \sim \log(1/\epsilon)$.

In AdS/CFT, Maldacena-Meyer (2020) generalizza: traversabilità richiede violazione locale della null energy condition (negative energy density), compatibile con Casimir energy o boundary conditions topologiche.

Implicazioni cosmologiche: wormhole traversabili potrebbero permettere correlazioni acausali limitate tra bolle in eternal inflation o tra rami Many-Worlds, o trasmissione quantistica tra universi entangled.

Confronto TET-CVTL: nel vacuum primordiale saturo di trefoil knots ($\trefoil$, $\lk{6}$), il braiding Ising eterno fornisce una forma persistente di «negative topological energy» (entanglement cosmico non locale). Questo rende gli ER bridges «semi-traversabili» topologicamente: l'informazione non viaggia causalmente attraverso il throat, ma l'invarianza $\lk{6}$ garantisce una connessione globale eterna tra rami/bolle, preservando l'unità del vacuum saturo senza violare la causalità macroscopica.

\section{Inflazione eterna e multiverso collegato da ER bridges}
L'inflazione eterna (Guth 1981; Linde 1986; Vilenkin 1983) è un'estensione dell'inflazione cosmica in cui l'espansione esponenziale non termina ovunque: regioni di false vacuum (alta energia) continuano a inflazionare indefinitamente, mentre fluttuazioni quantistiche nucleano bolle di true vacuum (pocket universes).

Nel modello caotico (Linde 1986), il campo inflatone $\phi$ evolve con potenziale $V(\phi)$ (es. $m^2\phi^2/2$), e fluttuazioni quantistiche $\delta\phi \sim H/(2\pi)$ possono spingere $\phi$ «uphill», prolungando l'inflazione localmente. Il criterio per eternal inflation è:
\begin{equation}
\Delta \phi_{qu} > 0.61 |\Delta \phi_{cl}| \quad \Rightarrow \quad \frac{H^2}{8\pi^2} > |\dot{\phi}_{cl}| H^{-1}.
\end{equation}

Questo genera un multiverso infinito fractal: volume inflating cresce esponenzialmente, bolle nucleano ad infinitum, ognuna con vacuum diverso (string landscape-like), leggi fisiche variabili, causally disconnected.

In TET-CVTL: il vacuum primordiale saturo di trefoil knots 
($\trefoil$, $\lk{6}$) rappresenta il false vacuum globale stabile 
sotto minimizzazione dell'azione Chern--Simons. 

Fluttuazioni combinate con braiding Ising eterno nucleano bolle 
tramite decay tunneling (instanton bounce), con $\lk{6}$ che funge 
da invariante topologico guidando le costanti fisiche 
($G \propto 1/L_k^2$, $\Lambda \propto L_k^{-1}$) e preservando 
la topologia tra le diverse bolle. 

Gli ER bridges collegano l'entanglement cosmico persistente tra 
le bolle, con il braiding eterno che funge da «ponte» topologico.

Confronto TET-CVTL: inflazione eterna è guidata dal vacuum trefoil saturo; bolle emergono come rami non intrecciati nel braid cosmico, con $\lk{6}$ come costante universale che unifica multiverso cosmologico e topologico.

\subsection{Inflazione eterna: Meccanismo e multiverso fractal}

L'inflazione eterna è un'estensione dell'inflazione cosmica in cui l'espansione esponenziale non termina ovunque: regioni di false vacuum (alta energia) continuano a inflazionare indefinitamente, mentre fluttuazioni quantistiche nucleano bolle di true vacuum (pocket universes) (Guth 1981; Linde 1986; Vilenkin 1983).

Nel modello caotico (Linde 1986), il campo inflatone $\phi$ evolve con potenziale $V(\phi)$ (es. $m^2\phi^2/2$ o $\lambda\phi^4/4$). Fluttuazioni quantistiche sono:
\begin{equation}
\delta \phi \sim \frac{H}{2\pi},
\end{equation}
dove $H = \sqrt{V/3M_P^2}$ è la Hubble durante inflazione.

Il criterio per eternal inflation è che la fluttuazione upward superi lo shift classico:
\begin{equation}
\Delta \phi_{qu} > 0.61 |\Delta \phi_{cl}| \quad \Rightarrow \quad \frac{H^2}{8\pi^2} > |\dot{\phi}_{cl}| H^{-1}.
\end{equation}

Questo genera un multiverso infinito fractal: volume inflating cresce esponenzialmente, bolle nucleano ad infinitum, ognuna con vacuum diverso (string landscape-like), leggi fisiche variabili, causally disconnected (no comunicazione superluminale tra bolle).

Nel framework TET-CVTL, il vacuum primordiale saturo di trefoil knots 
($\trefoil$, linking number invariante $\lk{6}$) rappresenta il false vacuum 
globale stabile, ottenuto dalla minimizzazione dell'azione Chern--Simons.

Le fluttuazioni, insieme al braiding Ising eterno, inducono la nucleazione 
di bolle attraverso processi di decay tunneling (instanton bounce). 
L'invariante $\lk{6}$ guida le costanti fondamentali 
($G \propto 1/L_k^2$, $\Lambda \propto L_k^{-1}$) e conserva la topologia 
tra le diverse bolle nucleate.

Gli Einstein--Rosen (ER) bridges connettono l'entanglement cosmico persistente 
tra le bolle, con il braiding eterno che funge da «ponte» topologico 
sottostante.

Confronto TET-CVTL: inflazione eterna è guidata dal vacuum trefoil saturo; bolle emergono come rami non intrecciati nel braid cosmico, con $\lk{6}$ come costante universale che unifica multiverso cosmologico, topologico e string landscape. La saturazione eterna impedisce terminazione globale dell'inflazione, mantenendo il multiverso fractal infinito e entangled.

\section{Loop Quantum Gravity e ER=EPR}

Loop Quantum Gravity (LQG) è una teoria di gravità quantistica background-independent: lo spazio-tempo è quantizzato in termini di spin networks (grafi con spin su edges e intertwiners su vertices). Gli operatori di area e volume sono discreti, con spettro quantizzato.

L'operatore di area associato a una superficie $\Sigma$ con puncture di spin $j$ è:
\begin{equation}
\hat{A} = 8\pi \gamma \ell_P^2 \sqrt{j(j+1)},
\end{equation}
dove $\gamma$ è il parametro di Immirzi (adimensionale), $\ell_P = \sqrt{\hbar G/c^3}$ è la lunghezza di Planck.

In LQG, la congettura ER=EPR è stata esplorata in termini di spin-network gluing (Rovelli-Vidotto 2014; Oriti et al. 2016; recentemente Tamburini et al. 2025): entanglement elementare tra due regioni corrisponde a un "minimal bridge" (puncture j=1/2 con gluing di spin-network). La lunghezza del throat è legata all'area operator:
\begin{equation}
A \propto \gamma \ell_P^2 \sqrt{j(j+1)}.
\end{equation}
Il parametro Immirzi $\gamma$ è stato derivato in alcuni modelli ER=EPR da entanglement/area increment minimo ($\log 2 / (\pi \sqrt{3})$).

Implicazioni per il multiverso: LQG prevede un Big Bounce invece di singolarità Big Bang, con transizione quantistica che genera rami cosmici. ER bridges quantistici in LQG potrebbero collegare universi post-bounce o bolle in eternal inflation.

Confronto con TET-CVTL: LQG ER=EPR estende TET-CVTL a una quantizzazione discreta dello spazio-tempo. Il vacuum primordiale saturo di trefoil knots ($\trefoil$, $\lk{6}$) fornisce l'invarianza topologica che guida la formazione di spin networks eterni. $\lk{6}$ agisce come invariante primordiale che seleziona configurazioni compatibili con minimal bridges (j=1/2) e braiding Ising eterno, unificando entanglement cosmico, geometria quantistica discreta e multiverso via quantum bridge (Big Bounce). Il linking number $\lk{6}$ sopravvive come costante topologica globale, collegando la rete di spin-network al vacuum eterno saturo.


\section{String landscape e multiverso}

La teoria delle stringhe descrive le particelle fondamentali come modi di vibrazione 
di stringhe unidimensionali in uno spaziotempo a 10 dimensioni 
(9 spaziali + 1 temporale). Le cinque teorie superstring consistenti 
(Type I, Type IIA, Type IIB, Heterotic SO(32) e Heterotic E$_8 \times$ E$_8$) 
sono collegate tra loro da una rete di dualità (T-dualità, S-dualità e U-dualità) 
e sono considerate limiti diversi di un'unica teoria sottostante in 11 dimensioni, 
nota come M-theory.

Compactificazione su varietà Ricci-flat (tipicamente Calabi-Yau 3-folds con holonomia SU(3)) genera effective 4D theories con gauge groups, chiralità e moduli scalari. Il **string landscape** emerge da miliardi di possibili compactificazioni + flux vacua (Giddings-Kachru-Polchinski 2002; Kachru-Kallosh-Linde-Trivedi 2003):
\begin{equation}
W = W_{\text{flux}} + W_{\text{non-pert}} = \int G_3 \wedge \Omega + e^{-T} + \dots,
\end{equation}
dove $T$ è il volume Kähler moduli, stabilizzato da flux superpotential + effetti non-perturbativi (KKLT, LVS scenarios).

Il numero stimato di vacua meta-stabili è compreso tra $10^{500}$ e $10^{1500}$, generando un multiverso di universi con costanti fisiche diverse (costante cosmologica $\Lambda$, masse, coupling costanti), con selezione anthropica per universi osservabili.

Nel contesto TET-CVTL: le dualità string e M-theory preservano l'invarianza topologica primordiale del vacuum trefoil eterno ($\trefoil$, $\lk{6}$). Il trefoil funge da topological seed che seleziona un subset di vacua landscape compatibili con saturazione eterna e braiding Ising persistente. $\lk{6}$ sopravvive come invariante globale sotto dualità (topologia 11D preservata), garantendo coerenza tra le diverse descrizioni duali (string/M-theory bulk vs CFT boundary) e tra vacua apparentemente disgiunti nel landscape.

Confronto TET-CVTL: il string landscape è ramificato in braid non intrecciati dal vacuum trefoil primordiale. $\lk{6}$ agisce come invariante primordiale che unifica il multiverso string con inflazione eterna e entanglement cosmico, fornendo un'ancora topologica che collega flux vacua, moduli stabilization e struttura gerarchica del multiverso.


\section{M-theory duality: Unificazione e web of dualities}

M-theory emerge come la teoria unificata che collega le cinque superstring theories consistenti in 10 dimensioni 
(Type I, Type IIA, Type IIB, Heterotic SO(32), Heterotic E$_8 \times$ E$_8$) 
e la supergravità 11-dimensionale (Witten, 1995). 

È definita come la struttura sottostante che unifica le cinque teorie attraverso una rete di dualità (web of dualities), 
in cui ciascuna teoria rappresenta un limite diverso della stessa M-theory.

Le dualità principali includono:
\begin{itemize}
    \item T-dualità: equivalenza tra teorie compatte su un cerchio di raggio $R$ e $1/R$ 
    (es. Type IIA $\leftrightarrow$ Type IIB; Heterotic SO(32) $\leftrightarrow$ Heterotic E$_8 \times$ E$_8$);
    
    \item S-dualità: inversione del coupling costante $g_s \to 1/g_s$ 
    (es. Type IIB self-duale sotto SL(2,$\mathbb{Z}$); Type I $\leftrightarrow$ Heterotic SO(32));
    
    \item U-dualità: combinazione di T- e S-dualità in dimensioni ridotte, 
    con gruppi discreti come SL(2,$\mathbb{Z}$), SO(4,4;$\mathbb{Z}$), SL(5,$\mathbb{Z}$), 
    SO(5,5;$\mathbb{Z}$), E$_{6(6)}$($\mathbb{Z}$), E$_{7(7)}$($\mathbb{Z}$), E$_{8(8)}$($\mathbb{Z}$) 
    (es. in 3 dimensioni emerge E$_{8(8)}$($\mathbb{Z}$));
    
    \item M-theory lift: Type IIA in 10D $\to$ M-theory su cerchio in 11D, 
    con M2-brane come stringhe wrapped e M5-brane come duali delle NS5-brane.
\end{itemize}

Il web of dualities forma una rete in cui ogni teoria è equivalente alle altre tramite trasformazioni dei moduli e del coupling. 
In compactificazioni su varietà Ricci-flat (Calabi--Yau, orbifold, manifold $G_2$), 
la U-dualità agisce sui moduli, vettori, tensori e cariche di brane, 
preservando l'azione effettiva e la superalgebra.

Confronto TET-CVTL: le dualità M-theory preservano l'invarianza topologica primordiale del vacuum trefoil eterno ($\trefoil$, $\lk{6}$). Il trefoil primordiale funge da topological seed che seleziona un subset di vacua compatibili con saturazione eterna e braiding Ising persistente. $\lk{6}$ sopravvive come invariante globale sotto dualità (topologia 11D preservata), garantendo coerenza tra le diverse descrizioni duali (string/M-theory bulk vs CFT boundary) e tra vacua apparentemente disgiunti nel landscape.

\subsection{U-dualità in M-theory}
La U-dualità è il gruppo di simmetria più ampio che emerge in M-theory 
dalla combinazione di T-dualità e S-dualità quando si riducono le dimensioni 
(dalla M-theory in 11D alle teorie effettive in $d \leq 9$ dimensioni). 

In $d$ dimensioni non-compactificate, il gruppo di U-dualità discreto è la 
split real form E$_{d(d)}$($\mathbb{Z}$) della supergravità massimale low-energy, 
che estende la T-dualità O($d,d$;$\mathbb{Z}$) e la S-dualità SL(2,$\mathbb{Z}$).

Esempi standard per compactificazioni toroidali:
\begin{itemize}
    \item $d=9$: SL(2,$\mathbb{Z}$) \quad (Type IIB S-dualità);
    \item $d=8$: SL(3,$\mathbb{Z}$) $\times$ SL(2,$\mathbb{Z}$);
    \item $d=7$: SL(5,$\mathbb{Z}$);
    \item $d=6$: SO(5,5;$\mathbb{Z}$);
    \item $d=5$: E$_{6(6)}$($\mathbb{Z}$);
    \item $d=4$: E$_{7(7)}$($\mathbb{Z}$);
    \item $d=3$: E$_{8(8)}$($\mathbb{Z}$).
\end{itemize}

Questi gruppi Cremmer--Julia governano le trasformazioni sui moduli scalari, 
sulle cariche di brane e sugli stati BPS, preservando l'azione effettiva.

La U-dualità agisce su moduli scalari, vettori, tensori e brane charges, preservando la consistenza del landscape: flux vacua e moduli stabilization devono essere invarianti sotto U-dualità per essere fisicamente significativi (Hull-Townsend 1995; Kachru-Kallosh-Linde-Trivedi 2003).

Confronto TET-CVTL: la U-dualità è analoga alla rete di braid non intrecciati – $\lk{6}$ del trefoil primordiale fornisce l'ancora topologica che unifica il multiverso M-theory con entanglement cosmico e inflazione eterna, collegando bulk geometry (wormhole ER) a boundary entanglement in un unico framework eterno.

\section{ABJM theory e sue implicazioni}

La teoria ABJM (Aharony-Bergman-Jafferis-Maldacena 2008) è una superconformal Chern-Simons theory in 3 dimensioni con gauge group $U(N)_k \times U(N)_{-k}$, supersimmetria $\mathcal{N}=6$ e materia in bifundamentale. È la duale AdS/CFT di M-theory su AdS$_4 \times S^7$ (o orbifold $S^7 / \mathbb{Z}_k$ per $k > 1$).

L'azione è data da termini Chern-Simons puri per le due connessioni $A$ e $\hat{A}$, più termini di materia scalare e fermionica:
\begin{equation}
S = \frac{k}{4\pi} \int \Tr \left( A \wedge dA + \frac{2}{3} A^3 \right) - \frac{k}{4\pi} \int \Tr \left( \hat{A} \wedge d\hat{A} + \frac{2}{3} \hat{A}^3 \right) + \text{matter terms}.
\end{equation}

Nel limite $k \to \infty$ (planar limit) e $N \to \infty$ con $k/N$ fissato, ABJM descrive M2-branes su orbifold e realizza una dualità con una teoria conforme 3D. Per valori piccoli di $k$ (es. $k=1,2$), emerge fisica anyonica non-Abeliana e fractional statistics.

Implicazioni per il multiverso: ABJM fornisce una realizzazione low-dimensional di AdS/CFT in 3+1 dimensioni, con dualità tra M-theory bulk e boundary CFT. È un ponte tra gravità quantistica 11D e teorie conformi 3D, utile per studiare entanglement cosmico in contesti di brane world e compactificazione.

Confronto TET-CVTL: ABJM realizza una dualità AdS/CFT in 3D+1 che preserva l'invarianza topologica primordiale del vacuum trefoil eterno ($\trefoil$, $\lk{6}$). Il braiding Ising eterno nel vacuum saturo emerge come anyonic statistics in ABJM (level $k=2$), mentre il linking number $\lk{6}$ funge da invariante globale che seleziona configurazioni compatibili con saturazione eterna. In questo modo, TET-CVTL unifica entanglement cosmico, braiding anyonico e dinamica conforme in un quadro multiversale topologico.


\section{AdS/CFT duality e sue implicazioni}

La corrispondenza AdS/CFT (Maldacena 1997) è una dualità tra una teoria gravitazionale in spazio anti-de Sitter (AdS) bulk e una teoria conforme dei campi (CFT) sul boundary. La versione classica è Type IIB string su AdS$_5 \times S^5$ duale a $\mathcal{N}=4$ SYM 4D.

La dualità afferma che:
\begin{equation}
Z_{\text{AdS gravity}}[g_{\mu\nu}, \phi, \dots] = Z_{\text{CFT}}[\text{operators on boundary}].
\end{equation}

L'entanglement entropy in CFT è data dalla formula Ryu-Takayanagi (2006):
\begin{equation}
S_A = \frac{\text{Area of minimal surface in bulk}}{4G},
\end{equation}
dove la superficie minimale è estesa nel bulk e ha boundary $\partial A$ sulla boundary CFT.

AdS/CFT è una realizzazione concreta di holografia: gravità in bulk emerge da una teoria quantistica senza gravità sul boundary. È valida in large-$N$, strong coupling, e ha estensioni a de Sitter (dS/CFT), flat space (flat holography) e low-dimensional cases (AdS$_2$/CFT$_1$, AdS$_3$/CFT$_2$).

Implicazioni per il multiverso: AdS/CFT fornisce un laboratorio per studiare entanglement cosmico, black hole information paradox e geometria emergente. In eternal inflation, potrebbe descrivere bolle come regioni boundary con entanglement globale.

Confronto TET-CVTL: AdS/CFT realizza una dualità tra bulk geometry (wormhole ER) e boundary entanglement (CFT), preservando l'invarianza topologica primordiale del vacuum trefoil eterno ($\trefoil$, $\lk{6}$). Il braiding Ising eterno nel vacuum saturo fornisce entanglement persistente analogo a quello boundary CFT, mentre $\lk{6}$ funge da invariante globale che seleziona configurazioni compatibili con saturazione eterna. In questo modo, TET-CVTL unifica entanglement cosmico, geometria olografica e multiverso ramificato in un quadro topologico profondo.


\section{SYK model e connessioni olografiche}

Il Sachdev-Ye-Kitaev (SYK) model (Sachdev-Ye 1986; Kitaev 2015) è un modello 0+1-dimensionale di $N$ fermioni Majorana con interazioni random all-to-all:
\begin{equation}
H = \sum_{1 \le i<j<k<l \le N} J_{ijkl} \psi_i \psi_j \psi_k \psi_l, \quad J_{ijkl} \sim \mathcal{N}(0, \frac{3! J^2}{N^3}).
\end{equation}

Nel large-$N$ limit ($N \to \infty$ con $J$ fissato), il SYK model realizza una dualità olografica con un black hole near-extremal in AdS$_2$/CFT$_1$ (Sachdev-Ye-Kitaev 2015; Maldacena-Stanford 2016). Ha le seguenti proprietà chiave:
\begin{itemize}
    \item caos massimo (Lyapunov exponent $\lambda_L = 2\pi T / \hbar$, saturazione bound chaos);
    \item assenza di quasiparticelle (no Fermi liquid);
    \item correlatori a due punti con scaling conforme $1/\sqrt{t}$;
    \item saturazione del bound di entanglement scrambling.
\end{itemize}

Il SYK model è un laboratorio ideale per studiare black hole interiors, informazione nel buco nero e entanglement in low-dimensional gravity.

Confronto TET-CVTL: il SYK model fornisce la dinamica chaotic maximally scrambling che collega entanglement cosmico a black hole interiors in un contesto olografico. Nel vacuum primordiale saturo di trefoil knots ($\trefoil$, $\lk{6}$), il braiding Ising eterno realizza un comportamento analogo a SYK-like (interazioni all-to-all topologiche), con $\lk{6}$ come invariante globale che stabilizza lo scrambling eterno. Questo unifica caos quantistico, entanglement persistente e topologia primordiale nel multiverso ramificato.

\subsection{Caos massimo nel SYK model}

Il Sachdev-Ye-Kitaev (SYK) model (Sachdev-Ye 1986; Kitaev 2015) è un modello 0+1-dimensionale di $N$ fermioni Majorana con interazioni random all-to-all:
\begin{equation}
H = \sum_{1 \le i<j<k<l \le N} J_{ijkl} \psi_i \psi_j \psi_k \psi_l, \quad J_{ijkl} \sim \mathcal{N}(0, \frac{3! J^2}{N^3}).
\end{equation}

Nel limite $N \to \infty$ (large-N), il SYK model realizza caos massimo: il Lyapunov exponent raggiunge il bound universale di caos:
\begin{equation}
\lambda_L = \frac{2\pi}{\beta \hbar} = 2\pi T,
\end{equation}
dove $T$ è la temperatura (Maldacena-Stanford 2016). Questo saturazione implica che il sistema è il più caotico possibile compatibile con unitarietà e locality.

Proprietà chiave del caos SYK:
\begin{itemize}
    \item assenza di quasiparticelle (no Fermi liquid);
    \item correlatori a due punti con scaling conforme $G(t) \sim 1/\sqrt{t}$;
    \item scrambling massimo dell'informazione (out-of-time-order correlators OTOC saturano bound);
    \item dualità olografica con black hole near-extremal in AdS$_2$/CFT$_1$.
\end{itemize}

Il caos SYK è un laboratorio ideale per studiare black hole interiors, paradox dell'informazione e entanglement scrambling in low-dimensional gravity.

Confronto TET-CVTL: il caos massimo nel SYK model corrisponde alla dinamica scrambling del braiding Ising eterno nel vacuum primordiale saturo di trefoil knots ($\trefoil$, $\lk{6}$). Il braiding Ising fornisce interazioni all-to-all topologiche che generano scrambling persistente senza decoerenza globale. $\lk{6}$ agisce come invariante primordiale che stabilizza il caos eterno, unificando scrambling quantistico, entanglement cosmico e struttura ramificata del multiverso.


\section{Black hole information paradox e sue risoluzioni}

Il black hole information paradox (Hawking 1975; Hawking 1976) nasce dal fatto che la radiazione Hawking è termica e apparentemente priva di informazione: un buco nero che evapora sembra perdere l'informazione iniziale, violando unitarietà quantistica.

La radiazione Hawking ha spettro termico:
\begin{equation}
T_H = \frac{\hbar c^3}{8\pi G M k_B},
\end{equation}
con entropia di Bekenstein-Hawking:
\begin{equation}
S_{BH} = \frac{k_B c^3 A}{4 \hbar G},
\end{equation}
dove $A = 4\pi r_s^2$ è l'area dell'orizzonte.

Il paradosso è: l'informazione iniziale (entanglement con l'esterno) sembra persa quando il buco nero evapora completamente.

Risoluzioni moderne:
\begin{itemize}
    \item Page curve (Page 1993): entropia di radiazione cresce fino a metà evaporazione, poi decresce (firewall o replica wormholes);
    \item Replica wormholes (Penington 2020; Almheiri et al. 2020): correzioni non-perturbative all'entanglement entropy in AdS/CFT;
    \item Island formula (Ryu-Takayanagi esteso): area di island in bulk dà entropia corretta;
    \item ER=EPR (Maldacena-Susskind 2013): informazione conservata in wormhole che collega interior a radiazione esterna.
\end{itemize}

Confronto TET-CVTL: il black hole information paradox è risolto nel vacuum primordiale saturo di trefoil knots ($\trefoil$, $\lk{6}$). Il braiding Ising eterno preserva entanglement cosmico persistente tra interior e exterior, mentre $\lk{6}$ funge da invariante topologico globale che impedisce perdita di informazione. L'informazione iniziale è codificata nell'invarianza topologica eterna del vacuum, unificando ER=EPR, replica wormholes e struttura ramificata del multiverso in un quadro topologico profondo.


\section{Conclusioni e prospettive}

Il framework TET-CVTL propone un paradigma in cui l'universo primordiale è un condensato eterno saturo di trefoil knots ($\trefoil$) con linking number invariante $\lk{6} = 6$, generatore di entanglement topologico persistente ($\gamma = \log \mathcal{D} \approx 0.658$) tramite braiding eterno di anyons Ising ($\mathrm{SU}(2)_2$). 

L'invariante $\lk{6}$ non è un parametro arbitrario, ma l'impronta digitale cosmologica fondamentale: stabilizza il vacuum pre-geometrico, vincola la nucleazione di bolle universi, preserva la coerenza topologica attraverso ER bridges e seleziona le costanti fisiche osservate ($G \propto 1/36$, $\Lambda \propto 1/6$) come residuo della protezione anyonica e della modularità cosmologica generalizzata.

TET-CVTL unifica topologia quantistica di campo (TQFT modulare, anyons non-Abelian), teoria dei nodi (Jones, HOMFLY, linking number) e cosmologia quantistica (ER=EPR esteso, eternal inflation, multi-Big Bang), offrendo un'origine topologica coerente per l'entanglement cosmico persistente e per l'assenza di decoerenza globale su scale cosmologiche.

Prospettive future includono:
\begin{itemize}
    \item simulazioni numeriche di correlatori di Wilson loops con $\lk{6}$ in Chern--Simons su lattice,
    \item estensione del modello a livelli $k > 2$ o anyons Fibonacci/metaplectic per descrivere transizioni di fase cosmologiche,
    \item ricerca di imprint di $\lk{6}$ e braiding eterno nelle anisotropie CMB, non-gaussianità primordiali e large-scale structure,
    \item sviluppo di una teoria effettiva low-energy che derivi la gravità classica da worldlines linked nel vacuum primordiale saturo.
\end{itemize}

Se confermata, TET-CVTL potrebbe rappresentare un paradigma unificato in cui la realtà fisica emerge da una rete topologica primordiale di entanglement persistente, con $\lk{6}$ come firma indelebile dell'origine cosmologica.



\section*{Note finali}

Questo lavoro è altamente speculativo e interdisciplinare. TET-CVTL non pretende di essere una teoria consolidata, ma un framework esplorativo che collega TQFT modulare, anyons non-Abelian, teoria dei nodi, ER=EPR e cosmologia quantistica in un tentativo di unificazione topologica.

Tutte le equazioni e i confronti sono basati su risultati noti della letteratura (citati), ma le estensioni cosmologiche (vacuum saturo con $\lk{6}$, braiding eterno come origine delle costanti, modularità deformata) sono proposte originali e non verificate sperimentalmente. 

Eventuali errori matematici o concettuali sono miei. 



\bibliographystyle{plain}
\bibliography{references}

@article{Witten1989,
  author = {Witten, Edward},
  title = {Quantum Field Theory and the Jones Polynomial},
  journal = {Communications in Mathematical Physics},
  volume = {121},
  number = {3},
  pages = {351--399},
  year = {1989},
  doi = {10.1007/BF01217730}
}

@article{MaldacenaSusskind2013,
  author = {Maldacena, Juan and Susskind, Leonard},
  title = {Cool horizons for entangled black holes},
  journal = {Fortschritte der Physik},
  volume = {61},
  number = {9},
  pages = {781--811},
  year = {2013},
  eprint = {1306.0533},
  archivePrefix = {arXiv},
  primaryClass = {hep-th},
  doi = {10.1002/prop.201300083}
}

@article{HartmanMaldacena2013,
  author = {Hartman, Thomas and Maldacena, Juan},
  title = {Entanglement Entropy from a Holographic Viewpoint},
  journal = {Journal of High Energy Physics},
  volume = {2013},
  number = {5},
  pages = {014},
  year = {2013},
  eprint = {1303.6955},
  archivePrefix = {arXiv},
  primaryClass = {hep-th},
  doi = {10.1007/JHEP05(2013)014}
}

@article{KitaevPreskill2006,
  author = {Kitaev, Alexei and Preskill, John},
  title = {Topological entanglement entropy},
  journal = {Physical Review Letters},
  volume = {96},
  number = {11},
  pages = {110404},
  year = {2006},
  eprint = {hep-th/0510092},
  archivePrefix = {arXiv},
  primaryClass = {hep-th},
  doi = {10.1103/PhysRevLett.96.110404}
}

@article{LevinWen2005,
  author = {Levin, Michael A. and Wen, Xiao-Gang},
  title = {String-net condensation: A physical mechanism for topological phases},
  journal = {Physical Review B},
  volume = {71},
  number = {4},
  pages = {045110},
  year = {2005},
  eprint = {cond-mat/0506438},
  archivePrefix = {arXiv},
  primaryClass = {cond-mat.str-el},
  doi = {10.1103/PhysRevB.71.045110}
}

@article{Tan2017,
  author = {Tan, Jonathan and others},
  title = {Topological entanglement from Rényi entropy in Chern-Simons theory},
  year = {2017},
  eprint = {1707.06629},
  archivePrefix = {arXiv},
  primaryClass = {hep-th}
}

@article{StanfordSusskind2014,
  author = {Stanford, Douglas and Susskind, Leonard},
  title = {Complexity and shock wave geometry},
  journal = {Physical Review D},
  volume = {90},
  number = {12},
  pages = {126007},
  year = {2014},
  eprint = {1401.5763},
  archivePrefix = {arXiv},
  primaryClass = {hep-th},
  doi = {10.1103/PhysRevD.90.126007}
}

@article{NayakEtAl2008,
  author = {Nayak, Chetan and Simon, Steven H. and Stern, Ady and Freedman, Michael and Sarma, S. Das},
  title = {Non-Abelian anyons and topological quantum computation},
  journal = {Reviews of Modern Physics},
  volume = {80},
  number = {3},
  pages = {1083--1159},
  year = {2008},
  doi = {10.1103/RevModPhys.80.1083}
}




\end{document}